\documentclass[parskip=full]{scrreprt}

\usepackage{helvet}
\usepackage[T1]{fontenc}
\usepackage[ngerman]{babel}
\usepackage[utf8]{inputenc}
\inputencoding{utf8}
\usepackage{hyperref}
\usepackage{mathtools}
\usepackage[xindy]{imakeidx}
\makeindex
\usepackage{enumitem}
\usepackage{graphicx}
\usepackage{cleveref}
\usepackage[autostyle=true,german=quotes]{csquotes}
\renewcommand{\familydefault}{\sfdefault}
\begin{document}


\title{Implementierungsbericht: RetroMachines, \\ RetroFactory}
\author{Luca Becker, Henrike Hardt,\\Larissa Schmid, Adrian Schulte,\\Maik Wiesner}
\date{1. Dezember 2014}
\maketitle
\thispagestyle{empty}

\clearpage

\thispagestyle{empty}
\tableofcontents
\thispagestyle{empty}

\clearpage
\setcounter{page}{1}

\chapter{Einleitung}

\chapter{Änderungen am Entwurf}

\chapter{Muss- und Wunschkriterien}

\section{Nicht umgesetzte Muss- und Wunschkriterien}

\begin{itemize}
	\item Auslieferung in mehreren Sprachen
	\item Challengemode mit Zeitdruck
	\item Werkstatt zum Umbauen von Maschinen
	\item Highscore-Tabelle
	\item Mehrere Spielmodi
	\item Begleitende Story
	\item Erweiternde Spielelemente
	\item Steuerung durch Wischgesten
\end{itemize}

\section{Umgesetzte Muss- und Wunschkriterien}

\begin{itemize}
	\item Pixelgrafik / Retrolook / reduzierter Farbraum
\end{itemize}

\chapter{Implementierungsplan und dessen Umsetzung}

\chapter{jUnit Tests}

\section{Package: models}

\subsection{GlobalVariablesTest}

Testcase zum Testen der GlobalVariables-Klasse

\begin{description}
\item[testCorrectValue]
	Testet, ob die GlobalVariables-Klasse den richtigen default Wert zurück gibt.
\item[testAssignValue]
	Testet, ob die GlobalVariables-Klasse einen Wert unter einem Key speichert.
\end{description}

\subsection{ProfileTest}

Testcase zum Testen der Profile-Klasse

\begin{description}
	\item[testCreateProfile] Testet, ob ein Profile erstellt werden kann und eine Statistik Instanz sowie Settings Instanz diesem zugewiesen werden kann.
	\item[testDBFetch] Testet, ob das Test-Profil aus der Datenbank geladen werden kann und dessen Werte stimmen.
	\item[testOverrideDB] Testet, ob ein Profil bearbeitet werden kann und diese Änderungen zurück in die Datenbank geschrieben werden.
	\item[testGetAllProfiles] Testet, ob die statische Methode \enquote{getAllProfiles} das richtige Profile und die richtige Anzahl an Profilen sich in dem Array befindet.
	\item[testGetAllProfilesWithCreate] Testet, ob die statische Methode \enquote{getAllProfiles} auch auf das Erstellen eines neuen Profiles reagiert.
	\item[testGetHashMap] Testet, ob die statische Methode \enquote{getProfileNameIdMap} die richtige Anzahl an Einträgen und den richtigen Eintrag beinhaltet.
\end{description}

\subsection{SettingTest}

Testcase zum Testen der Setting-Klasse.

\begin{itemize}
	\item[testCreateSetting] Testet, ob eine Einstellungs-Instanz erstellt werden kann und die richtigen Werte zugewiesen wurden.
	\item[testDBFetch] Testet, ob eine Einstellungs-Instanz basierend auf den Demoinformationen der Datenbank erstellt werden kann und über die richtigen Werte verfügt.
	\item[testWriteBack] Testet, ob eine Einstellungs-Instanz in der Lage ist geänderte Informationen zurück in die Datenbank zu schreiben.
\end{itemize}

\subsection{StatisticTest}

\begin{itemize}
	\item[testCreateStatistic] Testet, ob eine Statistik-Instanz erstellt werden kann und die richtigen Werte zugewiesen wurden.
	\item[testDBFetch] Testet, ob eine Statistik-Instanz basierend auf den Demoinformationen der Datenbank erstellt werden kann und über die richtigen Werte verfügt.
	\item[testWriteBack] Testet, ob eine Statistik-Instanz in der Lage ist geänderte Informationen zurück in die Datenbank zu schreiben.
\end{itemize}

\section{Package: controllers}

\subsection{GameControllerTest}

\begin{itemize}
	\item 
\end{itemize}

\subsection{ProfileControllerTest}

Testcase zum Testen der ProfileController-Klasse.

\begin{itemize}
	\item[testLoadLastProfile] Testet, ob der ProfileController in der Lage ist das zuletzt aktive Profile zu laden.
	\item[testRightLoading] Testet, ob der Controller das richtige Profil über die \enquote{loadLastProfile}-Methode geladen hat. Testet außerdem, ob das Laden scheitern kann.
	\item[testDoubleUserName] Testet, ob der Controller das Erstellen eines Profils mit bereits existierenden Namen erlaubt.
	\item[testMaximumProfiles] Testet, ob der Controller die Obergrenze an Profilen einhält.
	\item[testCreateDeleteProfile] Testet, ob der Controller ein Profil erstellt und die Vernichtung dieses Profils im Anschluss vollständig ist.
	\item[testChangeOnProfileCreate] Testet, ob das aktive Profil des Controllers im Anschluss an die Erstellung auf das erstellte Profil gewechselt wird.
	\item[testChangeProfile] Testet, ob der Controller beim Wechseln des Profils dieses in der Datenbank vermerkt.
	\item[testListener] Testet, ob der Controller korrekt die Klassen benachrichtigt, die sich bei ihm angemeldet haben. Dazu benutzt es seinen eigenen MockListener.
\end{itemize}



\end{document}
