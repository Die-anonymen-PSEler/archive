\documentclass{scrartcl}

\usepackage{helvet}
\usepackage[T1]{fontenc}
\usepackage[ngerman]{babel}
\usepackage[utf8]{inputenc}
\inputencoding{utf8}
\usepackage{hyperref}
\usepackage{mathtools}
\usepackage[xindy]{imakeidx}
\makeindex
\usepackage{enumitem}
\usepackage{graphicx}
\usepackage{cleveref}
\usepackage[autostyle=true,german=quotes]{csquotes}
\renewcommand{\familydefault}{\sfdefault}
\begin{document}


\title{Implementierungsbericht: RetroMachines, \\ RetroFactory}
\author{Luca Becker, Henrike Hardt,\\Larissa Schmid, Adrian Schulte,\\Maik Wiesner}
\date{1. Dezember 2014}
\maketitle
\thispagestyle{empty}

\clearpage

\thispagestyle{empty}
\tableofcontents
\thispagestyle{empty}

\clearpage
\setcounter{page}{1}

\section{Einleitung}

\section{Änderungen am Entwurf}

\section{Muss- und Wunschkriterien}

\subsection{Nicht umgesetzte Muss- und Wunschkriterien}

\begin{itemize}
	\item Auslieferung in mehreren Sprachen
	\item Challengemode mit Zeitdruck
	\item Werkstatt zum Umbauen von Maschinen
	\item Highscore-Tabelle
	\item Mehrere Spielmodi
	\item Begleitende Story
	\item Erweiternde Spielelemente
	\item Steuerung durch Wischgesten
\end{itemize}

\subsection{Umgesetzte Muss- und Wunschkriterien}

\begin{itemize}
	\item Pixelgrafik / Retrolook / reduzierter Farbraum
\end{itemize}

\section{Implementierungsplan und dessen Umsetzung}

\section{jUnit Tests}



\end{document}
