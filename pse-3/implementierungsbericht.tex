\documentclass[parskip=full]{scrreprt}

\usepackage{helvet}
\usepackage[T1]{fontenc}
\usepackage[ngerman]{babel}
\usepackage[utf8]{inputenc}
\inputencoding{utf8}
\usepackage{hyperref}
\usepackage{mathtools}
\usepackage[xindy]{imakeidx}
\makeindex
\usepackage{enumitem}
\usepackage{graphicx}
\usepackage{cleveref}
\usepackage[autostyle=true,german=quotes]{csquotes}
\renewcommand{\familydefault}{\sfdefault}
\begin{document}


\title{Implementierungsbericht: RetroMachines, \\ RetroFactory}
\author{Luca Becker, Henrike Hardt,\\Larissa Schmid, Adrian Schulte,\\Maik Wiesner}
\date{1. Dezember 2014}
\maketitle
\thispagestyle{empty}

\clearpage

\thispagestyle{empty}
\tableofcontents
\thispagestyle{empty}

\clearpage
\setcounter{page}{1}

\chapter{Einleitung}

"`Retro-Machines"' ist eine Lernapplikation, die Kindern ab dem Grundschulalter die Grundlagen der funktionalen Programmierung spielerisch näherbringen und sie so bereits früh für die Informatik begeistern soll. Wie in den vorangegangenen Phasen näher spezifiziert und skizziert wurde diese nun mithilfe von libgdx für Android implementiert.

Das vorliegende Dokument fasst hierbei den Ablauf der Implementierungsphase - Änderungen zum Entwurf, Umsetzungen der Muss- und Wunschkriterien, erste Testfälle mit jUnit sowie Zeitplanung unseres Teams - zusammen. Die Änderungen zum Entwurf sind hierbei zur leichteren Orientierung nach Packages sortiert und dokumentieren insbesondere die Veränderungen unserer Lambda-Datenstruktur.

\chapter{Änderungen am Entwurf}

Falls nicht anders angegeben handelt es sich bei den Methoden um public Methoden.

\section{Package: data}

\subsection{AssetManager}

\subsubsection{Hinzugefügte Methoden}
\begin{description}
	\item[Getter und Setter] Um auf die Spielelemente gezielt zugreifen zu können, haben wir uns entschieden, diese im AssetManager zu verwalten. Mit Übergabe der jeweiligen TiledMap werden daraufhin die Spielelemente ausgelesen und mithilfe der get-Methoden zurückgegeben.
\end{description}

\section{Package: models}

\subsection{Model}

\subsubsection{Hinzugefügte Methoden}
\begin{description}
	\item[destroy] Löscht den Eintrag, der aktuell durch das Model repräsentiert wird, durch das Model aus der Datenbank.
\end{description}

\subsubsection{Modifizierte Methoden}
\begin{description}
	\item[getStatement - \textit{protected}] Erstellt ein Statement, das zur Ausführung von Abfragen auf der Datenbank ausgeführt werden kann.
\end{description}

\subsection{Profile}

\subsubsection{Hinzugefügte Methoden}
\begin{description}
	\item[getAllProfiles] Erstellt ein String-Array, welches alle Profile enthält, die sich aktuell in der Datenbank befinden.
	\item[getProfileNameIdMap] Erstellt eine HashMap, welche als Abbildung von einem Profilnamen auf dessen Identifikationsnummer dient.
	\item[setStatistic] Ändert die Statistik des Profils zur übergebenen Instanz und speichert diese in der Datenbank
	
\end{description}

\subsection{Setting}

\subsubsection{Hinzugefügte Methoden}
\begin{description}
	\item[Konstruktor] Es wurde ein Konstruktor eingefügt, der das Erstellen eines Setting-Objektes mit Standardwerten erlaubt.
	
\end{description}

\subsection{Statistic}

\subsubsection{Hinzugefügte Methoden}
\begin{description}
	\item[Konstruktor] Es wurde ein Konstruktor eingefügt, der das Erstellen eines Statistic-Objektes mit Standardwerten erlaubt.
\end{description}

\section{Package: ui}

\subsection{RetroDialog}

Für verschiedene (Fehler)Dialoge, die dem Anwender während des Spieles angezeigt werden sollen - zum Beispiel bei Auswählen eines noch gesperrten Levels oder wenn er versucht, sein letztes Profil zu löschen - haben wir die Klasse RetroDialog angelegt. So sind diese leicht zu erstellen und von Design und Aufbau her einheitlich und leicht veränderbar.

\section{Package: game}

\subsection{RetroLevel und LevelBuilder}

\subsubsection{Aufgabe der Klasse}

\begin{description}
	\item RetroLevel gehört zum Model-Teil des Spiels und enthält alle wichtigen Informationen zum jeweiligen Level. Es wird mithilfe von der statischen, inneren Klasse LevelBuilder erstellt, die auch alle anderen Informationen wie die Map, die das jeweilige Level enthält, weitergibt. RetroLevel selber kann nicht direkt instanziiert werden durch einen privaten Konstruktor, um Inkonsistenzen zu vermeiden.
\end{description}

\subsubsection{Methoden von RetroLevel}

\begin{description}
	
	\item[isValidGameElementPosition] vergleicht, ob an der Stelle Platz ist, um ein GameElement abzulegen
	\item[placeGameElement] platziert ein GameElement
	\item[removeGameElement] entfernt ein GameElement
	\item[getTiles] gibt ein Array von Rechtecken um die angegebene Position zurück, aus denen man ablesen kann, ob dort freier Platz ist oder etwas steht auf der Map
	\item[allDepotsFilled] gibt zurück, ob alle Ablagen gefüllt sind
	\item[getGameElement] gibt ein GameElement an der übergebenen Position zurück
	\item[getMap] gibt die Map zurück, die das Level hält
	\item[getLambdaUtil] gibt das LambdaUtil, das das Level hält, zurück
	
\end{description}

\section{Package: game.gameelements}

\subsection{DepotElements}
Durch diese Klasse können wir nun auch die Ablagen für die Spielelemente direkt platzieren und müssen sie nicht mehr schon fix auf der TiledMap festlegen. Das macht die Platzierung und somit auch Evaluation einfacher.

\section{Package: game.controller}

\subsection{GameController}

\subsubsection{Hinzugefügte Methoden}
\begin{description}
	\item[update] Führt einen Schritt der Spielwelt aus.
\end{description}

\subsubsection{Entfernte Methoden}
\begin{description}
	\item[Evaluation] Aufgrund unseres Ziels des modularen Designs und der leichten Austauschbarkeit von Klassen bzw. Abläufen haben wir alle Methoden, die die Evaluation direkt betreffen, ausgelagert in die Lambda-Datenstruktur, welche nun also auch die Auswertung des Lambda-Terms vornimmt. Dies betrifft folgende Methoden:
		\begin{description}
			\item[evaluationClicked]
			\item[enterEvaluationScreen]
			\item[checkPlacementOfElements]
			\item[buildLambdaTree]
			\item[evaluate]
			\item[checkEvaluationResult]
			\item[betaReduction]
			\item[alphaConversion]
			\item[updateEvaluationScreen]
		\end{description}
	
\end{description}

\subsection{EvaluationController}
Anstelle der Methoden im GameController steht nun der EvaluationController.
	\subsubsection{Methoden}
	\begin{description} 
		\item[enterEvaluation] Zeigt dem Benutzer den Evaluationsbildschirm und holt sich den LambdaTree vom Level.
		\item[startEvaluation] startet die Evaluation
	\end{description}


\subsection{ProfileController}

\subsubsection{Hinzugefügte Methoden}
\begin{description}
	\item[deleteProfile]
\end{description}


\subsection{Lambda-Datenstruktur}

Im ursprünglichen Entwurf bestand der Baum, der den Lambda-Term darstellt, aus Instanzen der Klasse \textit{Vertex}. Diese hatten als Attribut den jeweiligen Typ (Applikation, Abstraktion, Variable). Wir haben uns entschieden, das zu ändern und stattdessen Polymorphie und dynamische Bindung zu verwenden. Die Klasse \textit{Vertex} ist nun abstrakt und hat die Subklassen \textit{Application}, \textit{Abstraction}, \textit{Variable} und \textit{Dummy}, welche für die Elemente des Lambda-Terms stehen. Dabei sind Instanzen von \textit{Dummy} Platzhalter (leere Knoten), mit denen der LevelTree zu Beginn initialisiert wird.
Falls das JSON des Levels nun zwar an sich eine gültige JSON-Syntax hat, diese aber nicht mehr als Lambda-Term korrekt ist, wird nun eine InvalidJsonException (auch in util.lambda) verwendet, um diesen Fall angemessen zu behandeln.

\chapter{Muss- und Wunschkriterien}

\section{Musskriterien}

Alle im Pflichtenheft bestimmten Musskriterien wurden implementiert. Um eine Übersicht zu gewährleisten, werden im Folgenden alle Kriterien samt Umsetzung aufgezählt.

\begin{description}
	\item Spielerisches Vermitteln des Lambda-Kalküls für Kinder ab acht Jahren: Durch das Jump'n'Run-Gameplay und die einfache Anordnung der Elemente im Spiel kann der Anwender schnell Erfolgserlebnisse sammeln. Dies geschieht durch interaktive Anordnung auf spannende Art und Weise.
	\item Das Spiel muss in der Simulationsumgebung von Android laufen: Dies wurde durch die Verwendung von libgdx sichergestellt. Ab API 17+ ist die Applikation auf allen Android-Geräten verfügbar.
	\item komplett über das simulierte Touchinterface bedienbar sein: Durch die Verwendung von vielfältigen Buttons ist es möglich, das Spiel komplett ohne Hardware-Tasten zu bedienen.
	\item eine modulare Architektur: Das Entwurfsmuster Model-View-Controller für das komplette Spiel sowie für die einzelnen Komponenten weitere Entwurfsmuster stellen die einfache Austauschbarkeit einzelner Komponenten sicher.
	\item mindestens fünf individuelle Level: Durch die derzeitige Bereitstellung von 7 Leveln haben wir dieses Musskriterium sogar ausgebaut.
	\item Tutorial-Level: Durch interaktive Dialoge während der ersten Level wird die einfache Einführung ins Prinzip des Spiels und des Lambda-Kalküls gewährleistet.
\end{description}

\section{Nicht umgesetzte Wunschkriterien}

\begin{description}
	\item Auslieferung in mehreren Sprachen
	\item Challengemode mit Zeitdruck
	\item Werkstatt zum Umbauen von Maschinen
	\item Highscore-Tabelle
	\item Mehrere Spielmodi
	\item Begleitende Story
	\item Steuerung durch Wischgesten
\end{description}

\section{Umgesetzte Wunschkriterien}

\begin{description}
	\item Mehrere Spielcharaktere
	\item Erweiternde Spielelemente (Einstellung des Steuerungsmodus: Rechts- oder Linkshänder)
	\item Pixelgrafik / Retrolook / reduzierter Farbraum
\end{description}

\chapter{Zeitlicher Ablauf}

\section{Adrian}

\begin{description}
	\item[23.01. - 29.01.15] Implementierung Screens/Menüs
	\item[30.01. - 02.02.] GameElements
	\item[23.01. - 29.01.] Grafiken
	\item[03.02. - 05.02] Level spielen
	\item[09.02 - 10.02] Tutorial
	\item[16.02. - 18.02] Lambda-Evaluation
	\item[16.02. - 20.02] Implementierungsbericht
\end{description}

\section{Henrike}

\begin{description}
\item[27.01.] Getter und Setter
\item[23.01. - 29.01.] Grafiken
\item[30.01. - 02.02.] Informationen aufrufen
\item[16.02. - 20.02] Implementierungsbericht
\end{description}

\section{Larissa}
\begin{description}
\item[30.01. - 02.02.] Starten des Spiels
\item[30.01. - 03.02.] Einstellungen bearbeiten
\item[03.02. - 05.02.] Level spielen
\item[30.01. - 03.02.] Statistiken
\item[16.02. - 20.02] Implementierungsbericht
\end{description}

\section{Luca}

\begin{description}
\item[21.01. - 26.01.]
Implementierung der Pakete \emph{data} und \emph{data.model}.
\item[28.01. - 30.01.]
Fehlerbehebung
\item[30.01. - 01.02.]
Implementierung des Pakets \emph{game.controllers}
\item[04.02. - 08.02.]
Implementierung der Klasse \emph{GameScreen} und Fehlerbehebung des Pakets \emph{game.controllers}
\item[08.02. - 13.02.]
Fehlerbehebung und Implementierungsbericht
\end{description}

\section{Maik}
\begin{description}
\item[30.01. - 02.02.] GameElements
\item[30.01. - 03.02.] Benutzer anlegen / bearbeiten
\item[16.02. - 18.02] Lambda-Evaluation
\item[28.02 - 03.03.] Präsentation
\item[16.02. - 20.02] Implementierungsbericht
\end{description}

\chapter{jUnit-Tests}

\section{Package: models}

\subsection{GlobalVariablesTest}

Testcase zum Testen der GlobalVariables-Klasse

\begin{description} 
\item[testCorrectValue]
	Testet, ob die GlobalVariables-Klasse den richtigen default-Wert zurückgibt.
\item[testAssignValue]
	Testet, ob die GlobalVariables-Klasse einen Wert unter einem Key speichert.
\end{description}

\subsection{ProfileTest}

Testcase zum Testen der Profile-Klasse

\begin{description}
	\item[testCreateProfile] Testet, ob ein Profile erstellt werden kann und diesem eine Statistic- sowie Settings-Instanz zugewiesen werden kann.
	\item[testDBFetch] Testet, ob das Test-Profil aus der Datenbank geladen werden kann und dessen Werte stimmen.
	\item[testOverrideDB] Testet, ob ein Profil bearbeitet werden kann und diese Änderungen zurück in die Datenbank geschrieben werden.
	\item[testGetAllProfiles] Testet, ob die statische Methode \enquote{getAllProfiles} das richtige Profile zurückgibt und ob sich die richtige Anzahl an Profilen in dem Array befindet.
	\item[testGetAllProfilesWithCreate] Testet, ob die statische Methode \enquote{getAllProfiles} auch auf das Erstellen eines neuen Profiles reagiert.
	\item[testGetHashMap] Testet, ob die statische Methode \enquote{getProfileNameIdMap} die richtige Anzahl an Einträgen und den richtigen Eintrag beinhaltet.
\end{description}

\subsection{SettingTest}

Testcase zum Testen der Setting-Klasse.

\begin{description}
	\item[testCreateSetting] Testet, ob eine Settings-Instanz erstellt werden kann und die richtigen Werte zugewiesen wurden.
	\item[testDBFetch] Testet, ob eine Settings-Instanz basierend auf den Demoinformationen der Datenbank erstellt werden kann und über die richtigen Werte verfügt.
	\item[testWriteBack] Testet, ob eine Settings-Instanz in der Lage ist, geänderte Informationen zurück in die Datenbank zu schreiben.
\end{description}

\subsection{StatisticTest}

\begin{description}
	\item[testCreateStatistic] Testet, ob eine Statistics-Instanz erstellt werden kann und die richtigen Werte zugewiesen wurden.
	\item[testDBFetch] Testet, ob eine Statistics-Instanz basierend auf den Demoinformationen der Datenbank erstellt werden kann und über die richtigen Werte verfügt.
	\item[testWriteBack] Testet, ob eine Statistics-Instanz in der Lage ist, geänderte Informationen zurück in die Datenbank zu schreiben.
\end{description}

\clearpage

\section{Package: controllers}

\subsection{GameControllerTest}

\begin{description}
	\item 
\end{description}

\subsection{ProfileControllerTest}

Testcase zum Testen der ProfileController-Klasse.

\begin{description}
	\item[testLoadLastProfile] Testet, ob der ProfileController in der Lage ist, das zuletzt aktive Profile zu laden.
	\item[testRightLoading] Testet, ob der Controller das richtige Profil über die \enquote{loadLastProfile}-Methode geladen hat. Testet außerdem, ob das Laden scheitern kann.
	\item[testDoubleUserName] Testet, ob der Controller das Erstellen eines Profils mit bereits existierenden Namen erlaubt.
	\item[testMaximumProfiles] Testet, ob der Controller die Obergrenze an Profilen einhält.
	\item[testCreateDeleteProfile] Testet, ob der Controller ein Profil erstellt und die Vernichtung dieses Profils im Anschluss vollständig ist.
	\item[testChangeOnProfileCreate] Testet, ob das aktive Profil des Controllers im Anschluss an die Erstellung auf das neu erstellte Profil gewechselt wird.
	\item[testChangeProfile] Testet, ob der Controller den Wechsel des aktiven Profils dieses in der Datenbank vermerkt.
	\item[testListener] Testet, ob der Controller die Klassen, die sich bei ihm angemeldet haben, korrekt benachrichtigt. Dazu benutzt es seinen eigenen MockListener.
\end{description}

\subsection{SettingController}

\begin{description}
	\item [testListener] Testet, ob ein MockListener durch den Controller über Änderungen an den Einstellungen benachrichtigt wird.
\end{description}

\section{Package: gameelements}

\subsection{RetroManTest}

\begin{description}
	\item[testJumpLock] Testet, ob ein erneuter Sprung möglich ist, nachdem RetroMan bereits zuvor gesprungen ist.
	\item[testFacing] Testet, ob die Figur nach links beziehungsweise nach rechts guckt.
	\item[testLeftMovement] Testet, ob die Figur korrekt nach links beschleunigt wird.
	\item[testRightMovement] Testet, ob die Figur korrekt nach rechts beschleunigt wird.
	\item[testJumpLockOnFalling] Testet, ob die Figur nicht springen kann, wenn sie fällt.
\end{description}


\end{document}
