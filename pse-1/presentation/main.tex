\documentclass[ngerman,hyperref={pdfpagelabels=false}]{beamer}

% -----------------------------------------------------------------------------

\graphicspath{{images/}}

% -----------------------------------------------------------------------------

\usetheme{KIT}

\setbeamercovered{transparent}
%\setbeamertemplate{enumerate items}[ball]

\newenvironment<>{KITtestblock}[2][]
{\begin{KITcolblock}<#1>{#2}{KITblack15}{KITblack50}}
{\end{KITcolblock}}

\usepackage[ngerman,english]{babel}
\usepackage[utf8]{inputenc}
\usepackage[TS1,T1]{fontenc}
\usepackage{array}
\usepackage{multicol}
\usepackage[absolute,overlay]{textpos}
\usepackage{beamerKITdefs}

\pdfpageattr {/Group << /S /Transparency /I true /CS /DeviceRGB>>}	%required to prevent color shifting withd transparent images


\title{Kolloqium Planungsphase}
\subtitle{Team B}

\author[Team B]{Team B}
\institute{Institut für Programmierparadigmen}

\TitleImage[width=\titleimagewd,height=\titleimageht]{titel}

\KITinstitute{Institut f\"ur Programmierparadigmen}
\KITfaculty{Fakult\"at f\"ur Informatik}

% -----------------------------------------------------------------------------

\begin{document}
\setlength\textheight{7cm} %required for correct vertical alignment, if [t] is not used as documentclass parameter


% title frame
\begin{frame}
  \maketitle
\end{frame}


\section{Vorstellung}
\begin{frame}
	\frametitle{Vorstellung Team B}
	\begin{itemize}
		\item Luca Becker, Henrike Hardt, Larissa Schmid, Adrian Schulte, Maik Wiesner
		\item PSE-Projekt: Android-App
	\end{itemize}
	\bigskip
\end{frame}


\section{Kriterien}
\begin{frame}
	\frametitle{Kriterien}
	\heading{Musskriterien}
	\begin{itemize}
		\item Spielerisches Vermitteln des $\lambda$-Kalküls
		\item Nach dem Wasserfall-Modell erarbeitetes Spiel
		\item mindestens fünf individuelle Level
	\end{itemize}
	\bigskip
\end{frame}

\section{Kriterien}
\begin{frame}
	\frametitle{Kriterien}
	\heading{Wunschkriterien}
	\begin{itemize}
		\item Retrolook
		\item Mehrere Spielmodi
		\item Begleitende Story
		\item Erweiternde Spiel- und Levelelemente
		\item Challengemode mit Zeitdruck
	\end{itemize}
\end{frame}

\section{Kriterien}
\begin{frame}
	\frametitle{Kriterien}
	\heading{Abgrenzungskriterien}
	\begin{itemize}
		\item Kein Mehrspieler Modus
		\item Keine Online-Anbindung
		\item Keine Kompatibilität mit anderen Geräten/Betriebssystemen ohne Emulation
	\end{itemize}
\end{frame}

\section{Spielidee}
\begin{frame}
	\frametitle{Spielidee}
	\begin{itemize}
		\item Jump'N'Run-Stil
		\item Spielerische Verdeutlichung $\lambda$-Kalküls durch Maschinen und Metallobjekt, die platziert werden müssen
	\end{itemize}	
\end{frame}

\section{Levelbeispiel}
\begin{frame}
	\frametitle{Levelbeispiel}
	\framesubtitle{$\lambda$-Kalkül}
	
	\[\lambda x.((\lambda f.f(f(x)))(\lambda y.y))(2)\]
\end{frame}

\section{Levelbeispiel}
\begin{frame}
	\frametitle{$\lambda$-Ausdruck in unserem Spiel}
	%bilder von Level Jeder Schritt
	%Bilder von Auswertung
	
\end{frame}



% blocks
\begin{frame}
	\frametitle{Blöcke}
	\framesubtitle{in bunt}

	\begin{KITalertblock}{alert}
	\begin{itemize}
	\item red
	\end{itemize}
	\end{KITalertblock}

	\bigskip
	\smallskip
	\pause

	\begin{KITinfoblock}{info}
	\begin{itemize}
		\item green
	\end{itemize}
	\end{KITinfoblock}
	
	\bigskip
	\smallskip
	\pause

	\begin{KITexampleblock}{example}
	\begin{itemize}
		\item blue
	\end{itemize}
	\end{KITexampleblock}

	\bigskip
	\smallskip
	\pause

	\begin{KITblock}{normal}
	\begin{itemize}
		\item gray
	\end{itemize}
	\end{KITblock}

\end{frame}


% blocks continued
\begin{frame}
	\frametitle{Blöcke}
	\framesubtitle{in bunt (2)}

	\begin{KITcolblock}{arbitrarily colored block}{KITgreen50}{KITred15}
	\begin{itemize}
		\item colorful
	\end{itemize}
	\end{KITcolblock}

	\bigskip
	\smallskip
	\pause

	\setboolean{KITblockborder}{false}
	\begin{KITblock}{no border}
	\begin{itemize}
		\item use \textbackslash setboolean\{KITblockborder\}\{true or false\} to switch block borders on or off
	\end{itemize}
	\end{KITblock}

	\bigskip
	\smallskip
	\pause	

	\heading{enumerate environment}
	\begin{enumerate}
		\item level 1
		\begin{enumerate}
			\item level 2
			\item level 2
			\begin{enumerate}
				\item level 3
				\item level 3
			\end{enumerate}
		\end{enumerate}
	\end{enumerate}

\end{frame}


% drawing area
\begin{frame}
	\frametitle{Area for Text}
	\framesubtitle{The big black box}

	\rule{\textwidth}{\textheight}
\end{frame}

\end{document}
