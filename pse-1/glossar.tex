\newglossaryentry{gls:anwender}
{
  name=Anwender,
  description={Beschreibt die Person die das Gerät in der Hand hält und bedient.}
}

\newglossaryentry{gls:lambdakalkuel}{
	name={Lambda Kalkül},
	description={Der $\lambda$-Kalkül ist eine formale Sprache, die im Allgemeinen dazu dient Funktionen zu definieren bzw. beschreiben.},
	plural={Lambda Kalküle}
}

\newglossaryentry{gls:android}{
	name={Android},
	description={Android ist sowohl ein Betriebssystem als auch eine Software-Plattform für mobile Geräte wie Smartphones, Mobiltelefone, Netbooks und Tablet-Computer,[3] die von der Open Handset Alliance (gegründet von Google) entwickelt wird.
	\\ Quelle: \href{http://de.wikipedia.org/wiki/Android_(Betriebssystem)}{Wikipedia, 25.11.2014}},
	plural={Android}
}

\newglossaryentry{gls:androidapp}{
	name={Android-App},
	description={Eine Android-App beschreibt ein Programm, das für den Einsatz auf der \gls{gls:android}-Plattform ausgelegt ist.},
	plural={Android-Apps}
}

\newglossaryentry{gls:simulationsumgebung}{
	name={Simulationsumgebung},
	description={Emuliert ein Android System auf einem Computer},
	plural={Simulationsumgebungen}
}

\newglossaryentry{gls:touchinterface}{
	name={Touchinterface},
	description={Ein Touchinterface beschriebt eine Technologie die es erlaubt mit einem Computer über haptischen Aktionen zu interagieren.},
	plural={Touchintefaces}
}

\newglossaryentry{gls:mvc}{
	name={Model-View-Controller},
	description={Der englischsprachige Begriff model view controller (MVC, englisch für Modell-Präsentation-Steuerung) ist ein Muster zur Strukturierung von Software-Entwicklung in die drei Einheiten Datenmodell (engl. model), Präsentation (engl. view) und Programmsteuerung (engl. controller).
	\\ Quelle: \href{http://de.wikipedia.org/wiki/Model_View_Controller}{Wikipedia, 25.11.2014}}
}

\newglossaryentry{gls:level}{
	name={Level},
	description={Ein Level (engl. ‚Ebene‘, ‚Stufe‘ von lateinisch libra = ‚Waage‘, ‚Gewogenes‘, ‚Gewichtseinheit‘, Artikel der oder das) ist in Computerspielen ein Abschnitt des Spiels, den der Spieler bewältigen muss, um in den nächsten Abschnitt zu gelangen.
	\\ Quelle: \href{http://de.wikipedia.org/wiki/Level_(Spielabschnitt)}{Wikipedia, 25.11.2014}}
}

\newglossaryentry{gls:onlinemodus}{
	name={Mehrspieler},
	description={Bei Computerspielen bezeichnet Mehrspieler (englisch multiplayer) einen Modus, bei dem man mit oder gegen andere Menschen spielt. Üblicherweise geschieht dies in einem Netzwerk (Internet, LAN-Party) von mehreren separaten Computern bzw. Spielkonsolen oder über die gemeinsame Nutzung nur eines einzelnen Geräts und Bildschirms mit mehreren Eingabemedien, zum Beispiel Gamepads. Auch rundenbasierte Spiele, in denen sich mehrere Spieler ein Gerät teilen und die Spielzüge abwechselnd ausführen, gehören dazu.
	\\ Quelle: \href{http://de.wikipedia.org/wiki/Mehrspieler}{Wikipedia, 25.11.2014}}
}