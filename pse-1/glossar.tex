\newglossaryentry{gls:anwender}
{
  name=Anwender,
  description={Beschreibt die Person die das Gerät in der Hand hält und bedient}
}

\newglossaryentry{gls:lambdakalkuel}{
	name={Lambda Kalkül},
	description={Der $\lambda$-Kalkül ist eine formale Sprache, die im Allgemeinen dazu dient Funktionen zu definieren bzw. beschreiben},
	plural={Lambda Kalküle}
}

\newglossaryentry{gls:android}{
	name={Android},
	description={Android ist sowohl ein Betriebssystem als auch eine Software-Plattform für mobile Geräte wie Smartphones, Mobiltelefone, Netbooks und Tablet-Computer, die von der Open Handset Alliance (gegründet von Google) entwickelt wird.
	\\ Quelle: \href{http://de.wikipedia.org/wiki/Android_(Betriebssystem)}{Wikipedia, 25.11.2014}},
	plural={Android}
}

\newglossaryentry{gls:androidapp}{
	name={Android-App},
	description={Eine Android-App beschreibt ein Programm, das für den Einsatz auf der \gls{gls:android}-Plattform ausgelegt ist},
	plural={Android-Apps}
}

\newglossaryentry{gls:simulationsumgebung}{
	name={Simulationsumgebung},
	description={Emuliert ein Betriebssytem auf einem anderen System},
	plural={Simulationsumgebungen}
}

\newglossaryentry{gls:touchinterface}{
	name={Touchinterface},
	description={Ein Touchinterface beschriebt eine Technologie die es erlaubt mit einem Computer über haptischen Aktionen zu interagieren},
	plural={Touchintefaces}
}

\newglossaryentry{gls:mvc}{
	name={Model-View-Controller},
	description={Der englischsprachige Begriff model view controller (MVC, englisch für Modell-Präsentation-Steuerung) ist ein Muster zur Strukturierung von Software-Entwicklung in die drei Einheiten Datenmodell (engl. model), Präsentation (engl. view) und Programmsteuerung (engl. controller).
	\\ Quelle: \href{http://de.wikipedia.org/wiki/Model_View_Controller}{Wikipedia, 25.11.2014}}
}

\newglossaryentry{gls:level}{
	name={Level},
	description={Ein Level (engl. ‚Ebene‘, ‚Stufe‘ von lateinisch libra = ‚Waage‘, ‚Gewogenes‘, ‚Gewichtseinheit‘, Artikel der oder das) ist in Computerspielen ein Abschnitt des Spiels, den der Spieler bewältigen muss, um in den nächsten Abschnitt zu gelangen.
	\\ Quelle: \href{http://de.wikipedia.org/wiki/Level_(Spielabschnitt)}{Wikipedia, 25.11.2014}},
	plural={Levels}
}

\newglossaryentry{gls:onlinemodus}{
	name={Mehrspieler},
	description={Bei Computerspielen bezeichnet Mehrspieler (englisch multiplayer) einen Modus, bei dem man mit oder gegen andere Menschen spielt. Üblicherweise geschieht dies in einem Netzwerk (Internet, LAN-Party) von mehreren separaten Computern bzw. Spielkonsolen oder über die gemeinsame Nutzung nur eines einzelnen Geräts und Bildschirms mit mehreren Eingabemedien, zum Beispiel Gamepads. Auch rundenbasierte Spiele, in denen sich mehrere Spieler ein Gerät teilen und die Spielzüge abwechselnd ausführen, gehören dazu.
	\\ Quelle: \href{http://de.wikipedia.org/wiki/Mehrspieler}{Wikipedia, 25.11.2014}}
}

\newglossaryentry{gls:tutorial}{
	name={Tutorial},
	description={Mit dem englischen Lehnwort Tutorial (lat. tueri „beschützen, bewahren, pflegen“) bezeichnet man im Computerjargon eine schriftliche oder filmische Gebrauchsanleitung oder auch einen Schnellkurs für Computerprogramme, in denen die Bedienung und die Funktionen anhand von (teils bebilderten) Beispielen Schritt für Schritt erklärt werden.
	\\ Quelle: \href{http://de.wikipedia.org/wiki/Tutorial}{Wikipedia, 26.11.2014}}
}

\newglossaryentry{gls:spieler}{
	name={Spieler},
	description={Analog zu \gls{gls:anwender}}
}

\newglossaryentry{gls:javadoc}{
	name={JavaDoc},
	description={Javadoc ist ein Software-Dokumentationswerkzeug, das aus Java-Quelltexten automatisch HTML-Dokumentationsdateien erstellt. Javadoc wurde ebenso wie Java von Sun Microsystems entwickelt und ist seit Version 2 ein Bestandteil des Java Development Kits.
	\\ Quelle: \href{http://de.wikipedia.org/wiki/Javadoc}{Wikipedia, 26.11.2014}}
}

\newglossaryentry{gls:junit}{
	name={jUnit},
	description={JUnit ist ein Framework zum Testen von Java-Programmen, das besonders für automatisierte Unit-Tests einzelner Units (Klassen oder Methoden) geeignet ist. Es basiert auf Konzepten, die ursprünglich unter dem Namen SUnit für Smalltalk entwickelt wurden.
	\\ Quelle: \href{http://de.wikipedia.org/wiki/JUnit}{Wikipedia, 26.11.2014}}
}

\newglossaryentry{gls:eclemma}{
	name={EclEMMA},
	description={Emma (Eigenschreibweise: EMMA) ist ein quelloffenes Werkzeug zur Messung der Testabdeckung in Java-Programmen. Dabei wird während der Ausführung einer Applikation gemessen, durch welche Klassen, Methoden, Blöcke und Zeilen Code die Abarbeitung lief.
	\\ Quelle: \href{http://de.wikipedia.org/wiki/Emma_(Software)}{Wikipedia, 26.11.2014}}
}

\newglossaryentry{gls:benutzerprofil}{
	name={Benutzerprofil},
	description={Ein Benutzerprofil ist in der Systemadministration eine Konfiguration eines Benutzerkontos bei einem Betriebssystem, einem Rechnernetz oder in einer Websiteverarbeitung. Diese Daten werden zentral in einem speziellen Verzeichnis (Sammelpunkt) verwaltet. Das Benutzerprofil von Computern enthält die Benutzerrechte eines oder mehrerer Benutzer sowie deren persönliche Einstellungen.
	\\ Quelle: \href{http://de.wikipedia.org/wiki/Benutzerprofil}{Wikipedia, 26.11.2014}},
	plural={Benutzerprofile}
}

\newglossaryentry{gls:hardware}{
	name={Hardware},
	description={Hardware (Abkürzung: HW) ist der Oberbegriff für die mechanische und elektronische Ausrüstung eines Systems, z. B. eines Computersystems. Er muss sich aber nicht ausschließlich auf Systeme mit einem Prozessor beziehen. Es können auch rein elektromechanische Geräte wie beispielsweise ein Treppenhauslicht-Automat sein. Ursprünglich ist das englische hardware ungefähr bedeutungsgleich mit „Eisenwaren” und wird heute im englischsprachigen Raum noch in diesem Sinne verwendet – also nicht nur für computer hardware.
	\\ Quelle: \href{http://de.wikipedia.org/wiki/Hardware}{Wikipedia, 26.11.2014}}
}

\newglossaryentry{gls:software}{
	name={Software},
	description={Software (dt. = weiche Ware [von] soft = leicht veränderbare Komponenten [...], Komplement zu ‚Hardware‘ für die physischen Komponenten) ist ein Sammelbegriff für Programme und die zugehörigen Daten. Sie kann als Beiwerk zusätzlich Bestandteile wie z. B. die Softwaredokumentation in der digitalen oder gedruckten Form eines Handbuchs enthalten.
	\\ Quelle: \href{http://de.wikipedia.org/wiki/Software}{Wikipedia, 26.11.2014}}
}

\newglossaryentry{gls:smartphone}{
	name={Smartphone},
	description={Ein Smartphone ist ein Mobiltelefon (umgangssprachlich Handy), das mehr Computer-Funktionalität und -konnektivität als ein herkömmliches fortschrittliches Mobiltelefon zur Verfügung stellt. Erste Smartphones vereinigten die Funktionen eines PDA bzw. Tabletcomputers mit der Funktionalität eines Mobiltelefons. Später wurde dem kompakten Gerät auch noch die Funktion eines transportablen Medienabspielgerätes, einer Digital- und Videokamera und eines GPS-Navigationsgeräts hinzugefügt.
	\\ Quelle: \href{http://de.wikipedia.org/wiki/Smartphone}{Wikipedia, 26.11.2014}}
}

\newglossaryentry{gls:tabletcomputer}{
	name={Tabletcomputer},
	description={Ein Tablet (englisch tablet ‚Schreibtafel‘, US-engl. tablet ‚Notizblock‘) oder Tabletcomputer, selten auch Flachrechner[1] ist ein tragbarer, flacher Computer in besonders leichter Ausführung mit einem Touchscreen, aber, anders als bei Notebooks, ohne ausklappbare mechanische Tastatur. Aufgrund der leichten Bauart und des berührungsempfindlichen Bildschirms zeichnen sich Tablets durch eine einfache Handhabung aus. Die Geräte ähneln in Leistungsumfang, Bedienung und Design modernen Smartphones und verwenden meist ursprünglich für Smartphones entwickelte Betriebssysteme wie Android oder Apple iOS. Wegen der Bildschirmtastatur, die nur bei Bedarf eingeblendet wird, eignen sich Tablets weniger gut für das Schreiben größerer Textmengen.
	\\ Quelle: \href{http://de.wikipedia.org/wiki/Tabletcomputer}{Wikipedia, 26.11.2014}}
}

\newglossaryentry{gls:achievement}{
	name={Achievement},
	description={In video gaming parlance, an achievement, also sometimes known as a trophy, badge, award, stamp, medal or challenge, is a meta-goal defined outside of a game's parameters. Unlike the systems of quests or levels that usually define the goals of a video game and have a direct effect on further gameplay, the management of achievements usually takes place outside the confines of the game environment and architecture.[1] Meeting the fulfillment conditions, and receiving recognition of fulfillment by the game, is referred to as unlocking the achievement. Despite the usual connotations of the term, unlocking an achievement does not generally pave the way for future actions with the same achievement.
	\\ Quelle: \href{http://de.wikipedia.org/wiki/Tabletcomputer}{Wikipedia, 26.11.2014}},
	plural={Achievements}
}





