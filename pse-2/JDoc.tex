\documentclass[11pt,a4paper]{report}
\usepackage{color}
\usepackage{ifthen}
\usepackage{makeidx}
\usepackage{ifpdf}
\usepackage[headings]{fullpage}
\ifpdf \usepackage[pdftex, pdfpagemode={UseOutlines},bookmarks,colorlinks,linkcolor={blue},plainpages=false,pdfpagelabels,citecolor={red},breaklinks=true]{hyperref}
  \usepackage[pdftex]{graphicx}
  \pdfcompresslevel=9
  \DeclareGraphicsRule{*}{mps}{*}{}
\else
  \usepackage[dvips]{graphicx}
\fi

\newcommand{\entityintro}[3]{%
  \hbox to \hsize{%
    \vbox{%
      \hbox to .2in{}%
    }%
    {\bf  #1}%
    \dotfill\pageref{#2}%
  }
  \makebox[\hsize]{%
    \parbox{.4in}{}%
    \parbox[l]{5in}{%
      \vspace{1mm}%
      #3%
      \vspace{1mm}%
    }%
  }%
}
\newcommand{\refdefined}[1]{
\expandafter\ifx\csname r@#1\endcsname\relax
\relax\else
{$($in \ref{#1}, page \pageref{#1}$)$}\fi}
\date{\today}
\chardef\textbackslash=`\\
\makeindex
\begin{document}
\sloppy
\addtocontents{toc}{\protect\markboth{Contents}{Contents}}
\tableofcontents
\chapter{Package com.retroMachines.game.Map}{
\label{com.retroMachines.game.Map}\hskip -.05in
\hbox to \hsize{\textit{ Package Contents\hfil Page}}
\vskip .13in
\hbox{{\bf  Classes}}
\entityintro{LambdaToMap}{com.retroMachines.game.Map.LambdaToMap}{}
\entityintro{Map}{com.retroMachines.game.Map.Map}{}
\vskip .1in
\vskip .1in
\section{\label{com.retroMachines.game.Map.LambdaToMap}\index{LambdaToMap}Class LambdaToMap}{
\vskip .1in 
\subsection{Declaration}{
\small public class LambdaToMap
\\ {\bf  extends} java.lang.Object
\refdefined{java.lang.Object}}
\subsection{Constructor summary}{
\begin{verse}
{\bf LambdaToMap()} \\
\end{verse}
}
\subsection{Method summary}{
\begin{verse}
{\bf getMapFromTerm(String)} \\
\end{verse}
}
\subsection{Constructors}{
\vskip -2em
\begin{itemize}
\item{ 
\index{LambdaToMap()}
{\bf  LambdaToMap}\\
\texttt{public\ {\bf  LambdaToMap}()
\label{com.retroMachines.game.Map.LambdaToMap()}}%end signature
}%end item
\end{itemize}
}
\subsection{Methods}{
\vskip -2em
\begin{itemize}
\item{ 
\index{getMapFromTerm(String)}
{\bf  getMapFromTerm}\\
\texttt{public Map\ {\bf  getMapFromTerm}(\texttt{java.lang.String} {\bf  lambda})
\label{com.retroMachines.game.Map.LambdaToMap.getMapFromTerm(java.lang.String)}}%end signature
}%end item
\end{itemize}
}
}
\section{\label{com.retroMachines.game.Map.Map}\index{Map}Class Map}{
\vskip .1in 
\subsection{Declaration}{
\small public class Map
\\ {\bf  extends} java.lang.Object
\refdefined{java.lang.Object}}
\subsection{Constructor summary}{
\begin{verse}
{\bf Map()} \\
\end{verse}
}
\subsection{Constructors}{
\vskip -2em
\begin{itemize}
\item{ 
\index{Map()}
{\bf  Map}\\
\texttt{public\ {\bf  Map}()
\label{com.retroMachines.game.Map.Map()}}%end signature
}%end item
\end{itemize}
}
}
}
\chapter{Package com.retroMachines.data}{
\label{com.retroMachines.data}\hskip -.05in
\hbox to \hsize{\textit{ Package Contents\hfil Page}}
\vskip .13in
\hbox{{\bf  Classes}}
\entityintro{AssetManager}{com.retroMachines.data.AssetManager}{}
\entityintro{RetroDatabase}{com.retroMachines.data.RetroDatabase}{}
\vskip .1in
\vskip .1in
\section{\label{com.retroMachines.data.AssetManager}\index{AssetManager}Class AssetManager}{
\vskip .1in 
\subsection{Declaration}{
\small public class AssetManager
\\ {\bf  extends} com.badlogic.gdx.assets.AssetManager
\refdefined{com.badlogic.gdx.assets.AssetManager}}
\subsection{Field summary}{
\begin{verse}
{\bf assetNames} Contains all file references to the Files to load.\\
{\bf manager} \\
{\bf menuSkin} \\
\end{verse}
}
\subsection{Constructor summary}{
\begin{verse}
{\bf AssetManager()} \\
\end{verse}
}
\subsection{Method summary}{
\begin{verse}
{\bf initialize()} loads all relevant objects into the cache of the game for flawless drawing\\
{\bf loadMap(int)} Loads a map from the Storage based on it's id\\
{\bf queueLoading()} \\
{\bf setMenuSkin()} \\
{\bf update()} \\
\end{verse}
}
\subsection{Fields}{
\begin{itemize}
\item{
\index{manager}
\label{com.retroMachines.data.AssetManager.manager}public static AssetManager {\bf  manager}}
\item{
\index{menuSkin}
\label{com.retroMachines.data.AssetManager.menuSkin}public static com.badlogic.gdx.scenes.scene2d.ui.Skin {\bf  menuSkin}}
\item{
\index{assetNames}
\label{com.retroMachines.data.AssetManager.assetNames}public static final java.lang.String {\bf  assetNames}\begin{itemize}
\item{\vskip -.9ex 
Contains all file references to the Files to load.}
\end{itemize}
}
\end{itemize}
}
\subsection{Constructors}{
\vskip -2em
\begin{itemize}
\item{ 
\index{AssetManager()}
{\bf  AssetManager}\\
\texttt{public\ {\bf  AssetManager}()
\label{com.retroMachines.data.AssetManager()}}%end signature
}%end item
\end{itemize}
}
\subsection{Methods}{
\vskip -2em
\begin{itemize}
\item{ 
\index{initialize()}
{\bf  initialize}\\
\texttt{public void\ {\bf  initialize}()
\label{com.retroMachines.data.AssetManager.initialize()}}%end signature
\begin{itemize}
\item{
{\bf  Description}

loads all relevant objects into the cache of the game for flawless drawing
}
\end{itemize}
}%end item
\item{ 
\index{loadMap(int)}
{\bf  loadMap}\\
\texttt{public static com.badlogic.gdx.maps.tiled.TiledMap\ {\bf  loadMap}(\texttt{int} {\bf  levelId})
\label{com.retroMachines.data.AssetManager.loadMap(int)}}%end signature
\begin{itemize}
\item{
{\bf  Description}

Loads a map from the Storage based on it's id
}
\item{
{\bf  Parameters}
  \begin{itemize}
   \item{
\texttt{levelId} -- the id of the map to load}
  \end{itemize}
}%end item
\item{{\bf  Returns} -- 
the map loaded as a TiledMap 
}%end item
\end{itemize}
}%end item
\item{ 
\index{queueLoading()}
{\bf  queueLoading}\\
\texttt{public static void\ {\bf  queueLoading}()
\label{com.retroMachines.data.AssetManager.queueLoading()}}%end signature
}%end item
\item{ 
\index{setMenuSkin()}
{\bf  setMenuSkin}\\
\texttt{public static void\ {\bf  setMenuSkin}()
\label{com.retroMachines.data.AssetManager.setMenuSkin()}}%end signature
}%end item
\item{ 
\index{update()}
{\bf  update}\\
\texttt{public synchronized boolean\ {\bf  update}()
\label{com.retroMachines.data.AssetManager.update()}}%end signature
}%end item
\end{itemize}
}
\subsection{Members inherited from class AssetManager }{
\texttt{com.badlogic.gdx.assets.AssetManager} {\small 
\refdefined{com.badlogic.gdx.assets.AssetManager}}
{\small 

\vskip -2em
\begin{itemize}
\item{\vskip -1.5ex 
\texttt{public synchronized void {\bf  clear}()
}%end signature
}%end item
\item{\vskip -1.5ex 
\texttt{public synchronized boolean {\bf  containsAsset}(\texttt{java.lang.Object} {\bf  arg0})
}%end signature
}%end item
\item{\vskip -1.5ex 
\texttt{public synchronized void {\bf  dispose}()
}%end signature
}%end item
\item{\vskip -1.5ex 
\texttt{public void {\bf  finishLoading}()
}%end signature
}%end item
\item{\vskip -1.5ex 
\texttt{public synchronized Object {\bf  get}(\texttt{AssetDescriptor} {\bf  arg0})
}%end signature
}%end item
\item{\vskip -1.5ex 
\texttt{public synchronized Object {\bf  get}(\texttt{java.lang.String} {\bf  arg0})
}%end signature
}%end item
\item{\vskip -1.5ex 
\texttt{public synchronized Object {\bf  get}(\texttt{java.lang.String} {\bf  arg0},
\texttt{java.lang.Class} {\bf  arg1})
}%end signature
}%end item
\item{\vskip -1.5ex 
\texttt{public synchronized Array {\bf  getAll}(\texttt{java.lang.Class} {\bf  arg0},
\texttt{com.badlogic.gdx.utils.Array} {\bf  arg1})
}%end signature
}%end item
\item{\vskip -1.5ex 
\texttt{public synchronized String {\bf  getAssetFileName}(\texttt{java.lang.Object} {\bf  arg0})
}%end signature
}%end item
\item{\vskip -1.5ex 
\texttt{public synchronized Array {\bf  getAssetNames}()
}%end signature
}%end item
\item{\vskip -1.5ex 
\texttt{public synchronized Class {\bf  getAssetType}(\texttt{java.lang.String} {\bf  arg0})
}%end signature
}%end item
\item{\vskip -1.5ex 
\texttt{public synchronized Array {\bf  getDependencies}(\texttt{java.lang.String} {\bf  arg0})
}%end signature
}%end item
\item{\vskip -1.5ex 
\texttt{public synchronized String {\bf  getDiagnostics}()
}%end signature
}%end item
\item{\vskip -1.5ex 
\texttt{public synchronized int {\bf  getLoadedAssets}()
}%end signature
}%end item
\item{\vskip -1.5ex 
\texttt{public AssetLoader {\bf  getLoader}(\texttt{java.lang.Class} {\bf  arg0})
}%end signature
}%end item
\item{\vskip -1.5ex 
\texttt{public AssetLoader {\bf  getLoader}(\texttt{java.lang.Class} {\bf  arg0},
\texttt{java.lang.String} {\bf  arg1})
}%end signature
}%end item
\item{\vskip -1.5ex 
\texttt{public Logger {\bf  getLogger}()
}%end signature
}%end item
\item{\vskip -1.5ex 
\texttt{public synchronized float {\bf  getProgress}()
}%end signature
}%end item
\item{\vskip -1.5ex 
\texttt{public synchronized int {\bf  getQueuedAssets}()
}%end signature
}%end item
\item{\vskip -1.5ex 
\texttt{public synchronized int {\bf  getReferenceCount}(\texttt{java.lang.String} {\bf  arg0})
}%end signature
}%end item
\item{\vskip -1.5ex 
\texttt{public synchronized boolean {\bf  isLoaded}(\texttt{java.lang.String} {\bf  arg0})
}%end signature
}%end item
\item{\vskip -1.5ex 
\texttt{public synchronized boolean {\bf  isLoaded}(\texttt{java.lang.String} {\bf  arg0},
\texttt{java.lang.Class} {\bf  arg1})
}%end signature
}%end item
\item{\vskip -1.5ex 
\texttt{public synchronized void {\bf  load}(\texttt{AssetDescriptor} {\bf  arg0})
}%end signature
}%end item
\item{\vskip -1.5ex 
\texttt{public synchronized void {\bf  load}(\texttt{java.lang.String} {\bf  arg0},
\texttt{java.lang.Class} {\bf  arg1})
}%end signature
}%end item
\item{\vskip -1.5ex 
\texttt{public synchronized void {\bf  load}(\texttt{java.lang.String} {\bf  arg0},
\texttt{java.lang.Class} {\bf  arg1},
\texttt{AssetLoaderParameters} {\bf  arg2})
}%end signature
}%end item
\item{\vskip -1.5ex 
\texttt{public synchronized void {\bf  setErrorListener}(\texttt{AssetErrorListener} {\bf  arg0})
}%end signature
}%end item
\item{\vskip -1.5ex 
\texttt{public synchronized void {\bf  setLoader}(\texttt{java.lang.Class} {\bf  arg0},
\texttt{loaders.AssetLoader} {\bf  arg1})
}%end signature
}%end item
\item{\vskip -1.5ex 
\texttt{public synchronized void {\bf  setLoader}(\texttt{java.lang.Class} {\bf  arg0},
\texttt{java.lang.String} {\bf  arg1},
\texttt{loaders.AssetLoader} {\bf  arg2})
}%end signature
}%end item
\item{\vskip -1.5ex 
\texttt{public void {\bf  setLogger}(\texttt{com.badlogic.gdx.utils.Logger} {\bf  arg0})
}%end signature
}%end item
\item{\vskip -1.5ex 
\texttt{public synchronized void {\bf  setReferenceCount}(\texttt{java.lang.String} {\bf  arg0},
\texttt{int} {\bf  arg1})
}%end signature
}%end item
\item{\vskip -1.5ex 
\texttt{public synchronized void {\bf  unload}(\texttt{java.lang.String} {\bf  arg0})
}%end signature
}%end item
\item{\vskip -1.5ex 
\texttt{public synchronized boolean {\bf  update}()
}%end signature
}%end item
\item{\vskip -1.5ex 
\texttt{public boolean {\bf  update}(\texttt{int} {\bf  arg0})
}%end signature
}%end item
\end{itemize}
}
}
\section{\label{com.retroMachines.data.RetroDatabase}\index{RetroDatabase}Class RetroDatabase}{
\vskip .1in 
\subsection{Declaration}{
\small public class RetroDatabase
\\ {\bf  extends} java.lang.Object
\refdefined{java.lang.Object}\\ {\bf  implements} 
com.badlogic.gdx.sql.Database}
\subsection{Method summary}{
\begin{verse}
{\bf closeDatabase()} \\
{\bf execSQL(String)} \\
{\bf getSingleton()} Returns the only copy of the database\\
{\bf openOrCreateDatabase()} \\
{\bf rawQuery(DatabaseCursor, String)} \\
{\bf rawQuery(String)} \\
{\bf setupDatabase()} \\
\end{verse}
}
\subsection{Methods}{
\vskip -2em
\begin{itemize}
\item{ 
\index{closeDatabase()}
{\bf  closeDatabase}\\
\texttt{ void\ {\bf  closeDatabase}() throws com.badlogic.gdx.sql.SQLiteGdxException
\label{com.retroMachines.data.RetroDatabase.closeDatabase()}}%end signature
}%end item
\item{ 
\index{execSQL(String)}
{\bf  execSQL}\\
\texttt{ void\ {\bf  execSQL}(\texttt{java.lang.String} {\bf  arg0}) throws com.badlogic.gdx.sql.SQLiteGdxException
\label{com.retroMachines.data.RetroDatabase.execSQL(java.lang.String)}}%end signature
}%end item
\item{ 
\index{getSingleton()}
{\bf  getSingleton}\\
\texttt{public static RetroDatabase\ {\bf  getSingleton}()
\label{com.retroMachines.data.RetroDatabase.getSingleton()}}%end signature
\begin{itemize}
\item{
{\bf  Description}

Returns the only copy of the database
}
\item{{\bf  Returns} -- 
database reference 
}%end item
\end{itemize}
}%end item
\item{ 
\index{openOrCreateDatabase()}
{\bf  openOrCreateDatabase}\\
\texttt{ void\ {\bf  openOrCreateDatabase}() throws com.badlogic.gdx.sql.SQLiteGdxException
\label{com.retroMachines.data.RetroDatabase.openOrCreateDatabase()}}%end signature
}%end item
\item{ 
\index{rawQuery(DatabaseCursor, String)}
{\bf  rawQuery}\\
\texttt{ com.badlogic.gdx.sql.DatabaseCursor\ {\bf  rawQuery}(\texttt{com.badlogic.gdx.sql.DatabaseCursor} {\bf  arg0},
\texttt{java.lang.String} {\bf  arg1}) throws com.badlogic.gdx.sql.SQLiteGdxException
\label{com.retroMachines.data.RetroDatabase.rawQuery(com.badlogic.gdx.sql.DatabaseCursor, java.lang.String)}}%end signature
}%end item
\item{ 
\index{rawQuery(String)}
{\bf  rawQuery}\\
\texttt{ com.badlogic.gdx.sql.DatabaseCursor\ {\bf  rawQuery}(\texttt{java.lang.String} {\bf  arg0}) throws com.badlogic.gdx.sql.SQLiteGdxException
\label{com.retroMachines.data.RetroDatabase.rawQuery(java.lang.String)}}%end signature
}%end item
\item{ 
\index{setupDatabase()}
{\bf  setupDatabase}\\
\texttt{ void\ {\bf  setupDatabase}()
\label{com.retroMachines.data.RetroDatabase.setupDatabase()}}%end signature
}%end item
\end{itemize}
}
}
}
\chapter{Package com.retroMachines.game}{
\label{com.retroMachines.game}\hskip -.05in
\hbox to \hsize{\textit{ Package Contents\hfil Page}}
\vskip .13in
\hbox{{\bf  Classes}}
\entityintro{Controller}{com.retroMachines.game.Controller}{}
\vskip .1in
\vskip .1in
\section{\label{com.retroMachines.game.Controller}\index{Controller}Class Controller}{
\vskip .1in 
\subsection{Declaration}{
\small public class Controller
\\ {\bf  extends} java.lang.Object
\refdefined{java.lang.Object}}
\subsection{Constructor summary}{
\begin{verse}
{\bf Controller()} \\
\end{verse}
}
\subsection{Constructors}{
\vskip -2em
\begin{itemize}
\item{ 
\index{Controller()}
{\bf  Controller}\\
\texttt{public\ {\bf  Controller}()
\label{com.retroMachines.game.Controller()}}%end signature
}%end item
\end{itemize}
}
}
}
\chapter{Package com.retroMachines.data.models}{
\label{com.retroMachines.data.models}\hskip -.05in
\hbox to \hsize{\textit{ Package Contents\hfil Page}}
\vskip .13in
\hbox{{\bf  Classes}}
\entityintro{Model}{com.retroMachines.data.models.Model}{}
\entityintro{Profile}{com.retroMachines.data.models.Profile}{Profile Class This class contains all information regarding the profile of a user It represents the model of each profile}
\entityintro{Setting}{com.retroMachines.data.models.Setting}{}
\entityintro{Statistic}{com.retroMachines.data.models.Statistic}{Statistics Class This class holds all information regarding the statistics of a single profile Every information stored within this class will be backed up to the sqlite database}
\vskip .1in
\vskip .1in
\section{\label{com.retroMachines.data.models.Model}\index{Model}Class Model}{
\vskip .1in 
\subsection{Declaration}{
\small public abstract class Model
\\ {\bf  extends} java.lang.Object
\refdefined{java.lang.Object}}
\subsection{All known subclasses}{Profile\small{\refdefined{com.retroMachines.data.models.Profile}}, Setting\small{\refdefined{com.retroMachines.data.models.Setting}}, Statistic\small{\refdefined{com.retroMachines.data.models.Statistic}}}
\subsection{Constructor summary}{
\begin{verse}
{\bf Model()} \\
\end{verse}
}
\subsection{Method summary}{
\begin{verse}
{\bf hasRecordInSQL()} Checks if the database has a possibly previous copy for the model\\
{\bf writeToSQL()} saves the model to the persistence background database\\
\end{verse}
}
\subsection{Constructors}{
\vskip -2em
\begin{itemize}
\item{ 
\index{Model()}
{\bf  Model}\\
\texttt{public\ {\bf  Model}()
\label{com.retroMachines.data.models.Model()}}%end signature
}%end item
\end{itemize}
}
\subsection{Methods}{
\vskip -2em
\begin{itemize}
\item{ 
\index{hasRecordInSQL()}
{\bf  hasRecordInSQL}\\
\texttt{public abstract boolean\ {\bf  hasRecordInSQL}()
\label{com.retroMachines.data.models.Model.hasRecordInSQL()}}%end signature
\begin{itemize}
\item{
{\bf  Description}

Checks if the database has a possibly previous copy for the model
}
\item{{\bf  Returns} -- 
true if a records exists; false otherwise 
}%end item
\end{itemize}
}%end item
\item{ 
\index{writeToSQL()}
{\bf  writeToSQL}\\
\texttt{public abstract void\ {\bf  writeToSQL}()
\label{com.retroMachines.data.models.Model.writeToSQL()}}%end signature
\begin{itemize}
\item{
{\bf  Description}

saves the model to the persistence background database
}
\end{itemize}
}%end item
\end{itemize}
}
}
\section{\label{com.retroMachines.data.models.Profile}\index{Profile}Class Profile}{
\vskip .1in 
Profile Class This class contains all information regarding the profile of a user It represents the model of each profile\vskip .1in 
\subsection{Declaration}{
\small public class Profile
\\ {\bf  extends} com.retroMachines.data.models.Model
\refdefined{com.retroMachines.data.models.Model}}
\subsection{Field summary}{
\begin{verse}
{\bf CREATE\_TABLE\_QUERY} a raw query that should be executed in case a table doesn't exist\\
{\bf DELETE\_TABLE\_QUERY\_PATTERN} a pattern (that should be formatted with printf or similar) that deletes a row within the TABLE\_NAME\\
{\bf INSERT\_TABLE\_QUERY\_PATTERN} a pattern (that should be formatted with printf or similar) that inserts a row within the TABLE\_NAME\\
{\bf TABLE\_NAME} the name of the table where the profiles are stored\\
{\bf UPDATE\_TABLE\_QUERY\_PATTERN} a pattern (that should be formatted with printf or similar) that updates a row within the TABLE\_NAME\\
\end{verse}
}
\subsection{Constructor summary}{
\begin{verse}
{\bf Profile(String, int, Setting, Statistic)} constructor for a new profile\\
\end{verse}
}
\subsection{Method summary}{
\begin{verse}
{\bf getProfileId()} \\
{\bf getProfileName()} \\
{\bf getSetting()} \\
{\bf hasRecordInSQL()} \\
{\bf setProfileId(int)} \\
{\bf setProfileName(String)} \\
{\bf setSetting(Setting)} \\
{\bf writeToSQL()} \\
\end{verse}
}
\subsection{Fields}{
\begin{itemize}
\item{
\index{TABLE\_NAME}
\label{com.retroMachines.data.models.Profile.TABLE_NAME}public static final java.lang.String {\bf  TABLE\_NAME}\begin{itemize}
\item{\vskip -.9ex 
the name of the table where the profiles are stored}
\end{itemize}
}
\item{
\index{CREATE\_TABLE\_QUERY}
\label{com.retroMachines.data.models.Profile.CREATE_TABLE_QUERY}public static final java.lang.String {\bf  CREATE\_TABLE\_QUERY}\begin{itemize}
\item{\vskip -.9ex 
a raw query that should be executed in case a table doesn't exist}
\end{itemize}
}
\item{
\index{UPDATE\_TABLE\_QUERY\_PATTERN}
\label{com.retroMachines.data.models.Profile.UPDATE_TABLE_QUERY_PATTERN}public static final java.lang.String {\bf  UPDATE\_TABLE\_QUERY\_PATTERN}\begin{itemize}
\item{\vskip -.9ex 
a pattern (that should be formatted with printf or similar) that updates a row within the TABLE\_NAME}
\end{itemize}
}
\item{
\index{DELETE\_TABLE\_QUERY\_PATTERN}
\label{com.retroMachines.data.models.Profile.DELETE_TABLE_QUERY_PATTERN}public static final java.lang.String {\bf  DELETE\_TABLE\_QUERY\_PATTERN}\begin{itemize}
\item{\vskip -.9ex 
a pattern (that should be formatted with printf or similar) that deletes a row within the TABLE\_NAME}
\end{itemize}
}
\item{
\index{INSERT\_TABLE\_QUERY\_PATTERN}
\label{com.retroMachines.data.models.Profile.INSERT_TABLE_QUERY_PATTERN}public static final java.lang.String {\bf  INSERT\_TABLE\_QUERY\_PATTERN}\begin{itemize}
\item{\vskip -.9ex 
a pattern (that should be formatted with printf or similar) that inserts a row within the TABLE\_NAME}
\end{itemize}
}
\end{itemize}
}
\subsection{Constructors}{
\vskip -2em
\begin{itemize}
\item{ 
\index{Profile(String, int, Setting, Statistic)}
{\bf  Profile}\\
\texttt{public\ {\bf  Profile}(\texttt{java.lang.String} {\bf  name},
\texttt{int} {\bf  profileId},
\texttt{Setting} {\bf  setting},
\texttt{Statistic} {\bf  statistic})
\label{com.retroMachines.data.models.Profile(java.lang.String, int, com.retroMachines.data.models.Setting, com.retroMachines.data.models.Statistic)}}%end signature
\begin{itemize}
\item{
{\bf  Description}

constructor for a new profile
}
\item{
{\bf  Parameters}
  \begin{itemize}
   \item{
\texttt{name} -- Name of the profile}
   \item{
\texttt{profileId} -- Id of the profile}
   \item{
\texttt{setting} -- settings of the profile}
   \item{
\texttt{statistic} -- statistics of the profile}
  \end{itemize}
}%end item
\end{itemize}
}%end item
\end{itemize}
}
\subsection{Methods}{
\vskip -2em
\begin{itemize}
\item{ 
\index{getProfileId()}
{\bf  getProfileId}\\
\texttt{public int\ {\bf  getProfileId}()
\label{com.retroMachines.data.models.Profile.getProfileId()}}%end signature
\begin{itemize}
\item{{\bf  Returns} -- 
the Id of the profile 
}%end item
\end{itemize}
}%end item
\item{ 
\index{getProfileName()}
{\bf  getProfileName}\\
\texttt{public java.lang.String\ {\bf  getProfileName}()
\label{com.retroMachines.data.models.Profile.getProfileName()}}%end signature
\begin{itemize}
\item{{\bf  Returns} -- 
the name of the profile 
}%end item
\end{itemize}
}%end item
\item{ 
\index{getSetting()}
{\bf  getSetting}\\
\texttt{public Setting\ {\bf  getSetting}()
\label{com.retroMachines.data.models.Profile.getSetting()}}%end signature
\begin{itemize}
\item{{\bf  Returns} -- 
the setting 
}%end item
\end{itemize}
}%end item
\item{ 
\index{hasRecordInSQL()}
{\bf  hasRecordInSQL}\\
\texttt{public abstract boolean\ {\bf  hasRecordInSQL}()
\label{com.retroMachines.data.models.Profile.hasRecordInSQL()}}%end signature
\begin{itemize}
\item{
{\bf  Description copied from Model{\small \refdefined{com.retroMachines.data.models.Model}} }

Checks if the database has a possibly previous copy for the model
}
\item{{\bf  Returns} -- 
true if a records exists; false otherwise 
}%end item
\end{itemize}
}%end item
\item{ 
\index{setProfileId(int)}
{\bf  setProfileId}\\
\texttt{public void\ {\bf  setProfileId}(\texttt{int} {\bf  profileId})
\label{com.retroMachines.data.models.Profile.setProfileId(int)}}%end signature
\begin{itemize}
\item{
{\bf  Parameters}
  \begin{itemize}
   \item{
\texttt{profileId} -- new Id of the profile}
  \end{itemize}
}%end item
\end{itemize}
}%end item
\item{ 
\index{setProfileName(String)}
{\bf  setProfileName}\\
\texttt{public void\ {\bf  setProfileName}(\texttt{java.lang.String} {\bf  profileName})
\label{com.retroMachines.data.models.Profile.setProfileName(java.lang.String)}}%end signature
\begin{itemize}
\item{
{\bf  Parameters}
  \begin{itemize}
   \item{
\texttt{profileName} -- new name of the profile}
  \end{itemize}
}%end item
\end{itemize}
}%end item
\item{ 
\index{setSetting(Setting)}
{\bf  setSetting}\\
\texttt{public void\ {\bf  setSetting}(\texttt{Setting} {\bf  setting})
\label{com.retroMachines.data.models.Profile.setSetting(com.retroMachines.data.models.Setting)}}%end signature
\begin{itemize}
\item{
{\bf  Parameters}
  \begin{itemize}
   \item{
\texttt{setting} -- the setting to set}
  \end{itemize}
}%end item
\end{itemize}
}%end item
\item{ 
\index{writeToSQL()}
{\bf  writeToSQL}\\
\texttt{public abstract void\ {\bf  writeToSQL}()
\label{com.retroMachines.data.models.Profile.writeToSQL()}}%end signature
\begin{itemize}
\item{
{\bf  Description copied from Model{\small \refdefined{com.retroMachines.data.models.Model}} }

saves the model to the persistence background database
}
\end{itemize}
}%end item
\end{itemize}
}
\subsection{Members inherited from class Model }{
\texttt{com.retroMachines.data.models.Model} {\small 
\refdefined{com.retroMachines.data.models.Model}}
{\small 

\vskip -2em
\begin{itemize}
\item{\vskip -1.5ex 
\texttt{public abstract boolean {\bf  hasRecordInSQL}()
}%end signature
}%end item
\item{\vskip -1.5ex 
\texttt{public abstract void {\bf  writeToSQL}()
}%end signature
}%end item
\end{itemize}
}
}
\section{\label{com.retroMachines.data.models.Setting}\index{Setting}Class Setting}{
\vskip .1in 
\subsection{Declaration}{
\small public class Setting
\\ {\bf  extends} com.retroMachines.data.models.Model
\refdefined{com.retroMachines.data.models.Model}}
\subsection{Field summary}{
\begin{verse}
{\bf CREATE\_TABLE\_QUERY} a raw query that should be executed in case a table doesn't exist\\
{\bf DELETE\_TABLE\_QUERY\_PATTERN} a pattern (that should be formatted with printf or similar) that deletes a row within the TABLE\_NAME\\
{\bf INSERT\_TABLE\_QUERY\_PATTERN} a pattern (that should be formatted with printf or similar) that inserts a row within the TABLE\_NAME\\
{\bf TABLE\_NAME} the name of the table where the settings are stored\\
{\bf UPDATE\_TABLE\_QUERY\_PATTERN} a pattern (that should be formatted with printf or similar) that updates a row within the TABLE\_NAME\\
\end{verse}
}
\subsection{Constructor summary}{
\begin{verse}
{\bf Setting(boolean, boolean, float)} \\
\end{verse}
}
\subsection{Method summary}{
\begin{verse}
{\bf getVolume()} \\
{\bf hasRecordInSQL()} \\
{\bf isLeftControl()} \\
{\bf isSoundOnOff()} \\
{\bf setLeftControl(boolean)} \\
{\bf setSoundOnOff(boolean)} \\
{\bf setVolume(float)} \\
{\bf writeToSQL()} \\
\end{verse}
}
\subsection{Fields}{
\begin{itemize}
\item{
\index{TABLE\_NAME}
\label{com.retroMachines.data.models.Setting.TABLE_NAME}public static final java.lang.String {\bf  TABLE\_NAME}\begin{itemize}
\item{\vskip -.9ex 
the name of the table where the settings are stored}
\end{itemize}
}
\item{
\index{CREATE\_TABLE\_QUERY}
\label{com.retroMachines.data.models.Setting.CREATE_TABLE_QUERY}public static final java.lang.String {\bf  CREATE\_TABLE\_QUERY}\begin{itemize}
\item{\vskip -.9ex 
a raw query that should be executed in case a table doesn't exist}
\end{itemize}
}
\item{
\index{UPDATE\_TABLE\_QUERY\_PATTERN}
\label{com.retroMachines.data.models.Setting.UPDATE_TABLE_QUERY_PATTERN}public static final java.lang.String {\bf  UPDATE\_TABLE\_QUERY\_PATTERN}\begin{itemize}
\item{\vskip -.9ex 
a pattern (that should be formatted with printf or similar) that updates a row within the TABLE\_NAME}
\end{itemize}
}
\item{
\index{DELETE\_TABLE\_QUERY\_PATTERN}
\label{com.retroMachines.data.models.Setting.DELETE_TABLE_QUERY_PATTERN}public static final java.lang.String {\bf  DELETE\_TABLE\_QUERY\_PATTERN}\begin{itemize}
\item{\vskip -.9ex 
a pattern (that should be formatted with printf or similar) that deletes a row within the TABLE\_NAME}
\end{itemize}
}
\item{
\index{INSERT\_TABLE\_QUERY\_PATTERN}
\label{com.retroMachines.data.models.Setting.INSERT_TABLE_QUERY_PATTERN}public static final java.lang.String {\bf  INSERT\_TABLE\_QUERY\_PATTERN}\begin{itemize}
\item{\vskip -.9ex 
a pattern (that should be formatted with printf or similar) that inserts a row within the TABLE\_NAME}
\end{itemize}
}
\end{itemize}
}
\subsection{Constructors}{
\vskip -2em
\begin{itemize}
\item{ 
\index{Setting(boolean, boolean, float)}
{\bf  Setting}\\
\texttt{public\ {\bf  Setting}(\texttt{boolean} {\bf  leftControl},
\texttt{boolean} {\bf  soundOnOff},
\texttt{float} {\bf  volume})
\label{com.retroMachines.data.models.Setting(boolean, boolean, float)}}%end signature
}%end item
\end{itemize}
}
\subsection{Methods}{
\vskip -2em
\begin{itemize}
\item{ 
\index{getVolume()}
{\bf  getVolume}\\
\texttt{public float\ {\bf  getVolume}()
\label{com.retroMachines.data.models.Setting.getVolume()}}%end signature
\begin{itemize}
\item{{\bf  Returns} -- 
the volume 
}%end item
\end{itemize}
}%end item
\item{ 
\index{hasRecordInSQL()}
{\bf  hasRecordInSQL}\\
\texttt{public abstract boolean\ {\bf  hasRecordInSQL}()
\label{com.retroMachines.data.models.Setting.hasRecordInSQL()}}%end signature
\begin{itemize}
\item{
{\bf  Description copied from Model{\small \refdefined{com.retroMachines.data.models.Model}} }

Checks if the database has a possibly previous copy for the model
}
\item{{\bf  Returns} -- 
true if a records exists; false otherwise 
}%end item
\end{itemize}
}%end item
\item{ 
\index{isLeftControl()}
{\bf  isLeftControl}\\
\texttt{public boolean\ {\bf  isLeftControl}()
\label{com.retroMachines.data.models.Setting.isLeftControl()}}%end signature
\begin{itemize}
\item{{\bf  Returns} -- 
the leftControl 
}%end item
\end{itemize}
}%end item
\item{ 
\index{isSoundOnOff()}
{\bf  isSoundOnOff}\\
\texttt{public boolean\ {\bf  isSoundOnOff}()
\label{com.retroMachines.data.models.Setting.isSoundOnOff()}}%end signature
\begin{itemize}
\item{{\bf  Returns} -- 
the soundOnOff 
}%end item
\end{itemize}
}%end item
\item{ 
\index{setLeftControl(boolean)}
{\bf  setLeftControl}\\
\texttt{public void\ {\bf  setLeftControl}(\texttt{boolean} {\bf  leftControl})
\label{com.retroMachines.data.models.Setting.setLeftControl(boolean)}}%end signature
\begin{itemize}
\item{
{\bf  Parameters}
  \begin{itemize}
   \item{
\texttt{leftControl} -- the leftControl to set}
  \end{itemize}
}%end item
\end{itemize}
}%end item
\item{ 
\index{setSoundOnOff(boolean)}
{\bf  setSoundOnOff}\\
\texttt{public void\ {\bf  setSoundOnOff}(\texttt{boolean} {\bf  soundOnOff})
\label{com.retroMachines.data.models.Setting.setSoundOnOff(boolean)}}%end signature
\begin{itemize}
\item{
{\bf  Parameters}
  \begin{itemize}
   \item{
\texttt{soundOnOff} -- the soundOnOff to set}
  \end{itemize}
}%end item
\end{itemize}
}%end item
\item{ 
\index{setVolume(float)}
{\bf  setVolume}\\
\texttt{public void\ {\bf  setVolume}(\texttt{float} {\bf  volume})
\label{com.retroMachines.data.models.Setting.setVolume(float)}}%end signature
\begin{itemize}
\item{
{\bf  Parameters}
  \begin{itemize}
   \item{
\texttt{volume} -- the volume to set}
  \end{itemize}
}%end item
\end{itemize}
}%end item
\item{ 
\index{writeToSQL()}
{\bf  writeToSQL}\\
\texttt{public abstract void\ {\bf  writeToSQL}()
\label{com.retroMachines.data.models.Setting.writeToSQL()}}%end signature
\begin{itemize}
\item{
{\bf  Description copied from Model{\small \refdefined{com.retroMachines.data.models.Model}} }

saves the model to the persistence background database
}
\end{itemize}
}%end item
\end{itemize}
}
\subsection{Members inherited from class Model }{
\texttt{com.retroMachines.data.models.Model} {\small 
\refdefined{com.retroMachines.data.models.Model}}
{\small 

\vskip -2em
\begin{itemize}
\item{\vskip -1.5ex 
\texttt{public abstract boolean {\bf  hasRecordInSQL}()
}%end signature
}%end item
\item{\vskip -1.5ex 
\texttt{public abstract void {\bf  writeToSQL}()
}%end signature
}%end item
\end{itemize}
}
}
\section{\label{com.retroMachines.data.models.Statistic}\index{Statistic}Class Statistic}{
\vskip .1in 
Statistics Class This class holds all information regarding the statistics of a single profile Every information stored within this class will be backed up to the sqlite database\vskip .1in 
\subsection{Declaration}{
\small public class Statistic
\\ {\bf  extends} com.retroMachines.data.models.Model
\refdefined{com.retroMachines.data.models.Model}}
\subsection{Field summary}{
\begin{verse}
{\bf CREATE\_TABLE\_QUERY} a raw query that should be executed in case a table doesn't exist\\
{\bf DELETE\_TABLE\_QUERY\_PATTERN} a pattern (that should be formatted with printf or similar) that deletes a row within the TABLE\_NAME\\
{\bf INSERT\_TABLE\_QUERY\_PATTERN} a pattern (that should be formatted with printf or similar) that inserts a row within the TABLE\_NAME\\
{\bf TABLE\_NAME} the name of the table where the statistics are stored\\
{\bf UPDATE\_TABLE\_QUERY\_PATTERN} a pattern (that should be formatted with printf or similar) that updates a row within the TABLE\_NAME\\
\end{verse}
}
\subsection{Constructor summary}{
\begin{verse}
{\bf Statistic(int, int, int)} creates a new instance of Statistic and assigns all the variables to the instance\\
\end{verse}
}
\subsection{Method summary}{
\begin{verse}
{\bf getLevelsComplete()} \\
{\bf getPlaytime()} \\
{\bf getStepCounter()} \\
{\bf hasRecordInSQL()} \\
{\bf setLevelsComplete(int)} \\
{\bf setPlaytime(int)} \\
{\bf setStepCounter(int)} \\
{\bf writeToSQL()} \\
\end{verse}
}
\subsection{Fields}{
\begin{itemize}
\item{
\index{TABLE\_NAME}
\label{com.retroMachines.data.models.Statistic.TABLE_NAME}public static final java.lang.String {\bf  TABLE\_NAME}\begin{itemize}
\item{\vskip -.9ex 
the name of the table where the statistics are stored}
\end{itemize}
}
\item{
\index{CREATE\_TABLE\_QUERY}
\label{com.retroMachines.data.models.Statistic.CREATE_TABLE_QUERY}public static final java.lang.String {\bf  CREATE\_TABLE\_QUERY}\begin{itemize}
\item{\vskip -.9ex 
a raw query that should be executed in case a table doesn't exist}
\end{itemize}
}
\item{
\index{UPDATE\_TABLE\_QUERY\_PATTERN}
\label{com.retroMachines.data.models.Statistic.UPDATE_TABLE_QUERY_PATTERN}public static final java.lang.String {\bf  UPDATE\_TABLE\_QUERY\_PATTERN}\begin{itemize}
\item{\vskip -.9ex 
a pattern (that should be formatted with printf or similar) that updates a row within the TABLE\_NAME}
\end{itemize}
}
\item{
\index{DELETE\_TABLE\_QUERY\_PATTERN}
\label{com.retroMachines.data.models.Statistic.DELETE_TABLE_QUERY_PATTERN}public static final java.lang.String {\bf  DELETE\_TABLE\_QUERY\_PATTERN}\begin{itemize}
\item{\vskip -.9ex 
a pattern (that should be formatted with printf or similar) that deletes a row within the TABLE\_NAME}
\end{itemize}
}
\item{
\index{INSERT\_TABLE\_QUERY\_PATTERN}
\label{com.retroMachines.data.models.Statistic.INSERT_TABLE_QUERY_PATTERN}public static final java.lang.String {\bf  INSERT\_TABLE\_QUERY\_PATTERN}\begin{itemize}
\item{\vskip -.9ex 
a pattern (that should be formatted with printf or similar) that inserts a row within the TABLE\_NAME}
\end{itemize}
}
\end{itemize}
}
\subsection{Constructors}{
\vskip -2em
\begin{itemize}
\item{ 
\index{Statistic(int, int, int)}
{\bf  Statistic}\\
\texttt{public\ {\bf  Statistic}(\texttt{int} {\bf  playtime},
\texttt{int} {\bf  levelsComplete},
\texttt{int} {\bf  stepCounter})
\label{com.retroMachines.data.models.Statistic(int, int, int)}}%end signature
\begin{itemize}
\item{
{\bf  Description}

creates a new instance of Statistic and assigns all the variables to the instance
}
\item{
{\bf  Parameters}
  \begin{itemize}
   \item{
\texttt{playtime} -- }
   \item{
\texttt{levelsComplete} -- }
   \item{
\texttt{stepCounter} -- }
  \end{itemize}
}%end item
\end{itemize}
}%end item
\end{itemize}
}
\subsection{Methods}{
\vskip -2em
\begin{itemize}
\item{ 
\index{getLevelsComplete()}
{\bf  getLevelsComplete}\\
\texttt{public int\ {\bf  getLevelsComplete}()
\label{com.retroMachines.data.models.Statistic.getLevelsComplete()}}%end signature
}%end item
\item{ 
\index{getPlaytime()}
{\bf  getPlaytime}\\
\texttt{public int\ {\bf  getPlaytime}()
\label{com.retroMachines.data.models.Statistic.getPlaytime()}}%end signature
}%end item
\item{ 
\index{getStepCounter()}
{\bf  getStepCounter}\\
\texttt{public int\ {\bf  getStepCounter}()
\label{com.retroMachines.data.models.Statistic.getStepCounter()}}%end signature
}%end item
\item{ 
\index{hasRecordInSQL()}
{\bf  hasRecordInSQL}\\
\texttt{public abstract boolean\ {\bf  hasRecordInSQL}()
\label{com.retroMachines.data.models.Statistic.hasRecordInSQL()}}%end signature
\begin{itemize}
\item{
{\bf  Description copied from Model{\small \refdefined{com.retroMachines.data.models.Model}} }

Checks if the database has a possibly previous copy for the model
}
\item{{\bf  Returns} -- 
true if a records exists; false otherwise 
}%end item
\end{itemize}
}%end item
\item{ 
\index{setLevelsComplete(int)}
{\bf  setLevelsComplete}\\
\texttt{public void\ {\bf  setLevelsComplete}(\texttt{int} {\bf  levelsComplete})
\label{com.retroMachines.data.models.Statistic.setLevelsComplete(int)}}%end signature
}%end item
\item{ 
\index{setPlaytime(int)}
{\bf  setPlaytime}\\
\texttt{public void\ {\bf  setPlaytime}(\texttt{int} {\bf  playtime})
\label{com.retroMachines.data.models.Statistic.setPlaytime(int)}}%end signature
}%end item
\item{ 
\index{setStepCounter(int)}
{\bf  setStepCounter}\\
\texttt{public void\ {\bf  setStepCounter}(\texttt{int} {\bf  stepCounter})
\label{com.retroMachines.data.models.Statistic.setStepCounter(int)}}%end signature
}%end item
\item{ 
\index{writeToSQL()}
{\bf  writeToSQL}\\
\texttt{public abstract void\ {\bf  writeToSQL}()
\label{com.retroMachines.data.models.Statistic.writeToSQL()}}%end signature
\begin{itemize}
\item{
{\bf  Description copied from Model{\small \refdefined{com.retroMachines.data.models.Model}} }

saves the model to the persistence background database
}
\end{itemize}
}%end item
\end{itemize}
}
\subsection{Members inherited from class Model }{
\texttt{com.retroMachines.data.models.Model} {\small 
\refdefined{com.retroMachines.data.models.Model}}
{\small 

\vskip -2em
\begin{itemize}
\item{\vskip -1.5ex 
\texttt{public abstract boolean {\bf  hasRecordInSQL}()
}%end signature
}%end item
\item{\vskip -1.5ex 
\texttt{public abstract void {\bf  writeToSQL}()
}%end signature
}%end item
\end{itemize}
}
}
}
\chapter{Package com.retroMachines.ui.screens}{
\label{com.retroMachines.ui.screens}\hskip -.05in
\hbox to \hsize{\textit{ Package Contents\hfil Page}}
\vskip .13in
\hbox{{\bf  Classes}}
\entityintro{AbstractScreen}{com.retroMachines.ui.screens.AbstractScreen}{Abstrakte Screen Klasse auf die alle Anzeigen des Spiels handhabt.}
\vskip .1in
\vskip .1in
\section{\label{com.retroMachines.ui.screens.AbstractScreen}\index{AbstractScreen}Class AbstractScreen}{
\vskip .1in 
Abstrakte Screen Klasse auf die alle Anzeigen des Spiels handhabt.\vskip .1in 
\subsection{Declaration}{
\small public abstract class AbstractScreen
\\ {\bf  extends} java.lang.Object
\refdefined{java.lang.Object}\\ {\bf  implements} 
com.badlogic.gdx.Screen}
\subsection{All known subclasses}{ProfileMenuScreen\small{\refdefined{com.retroMachines.ui.screens.menus.ProfileMenuScreen}}, AboutMenuScreen\small{\refdefined{com.retroMachines.ui.screens.menus.AboutMenuScreen}}, SettingsMenuScreen\small{\refdefined{com.retroMachines.ui.screens.menus.SettingsMenuScreen}}, CreateProfileMenuScreen\small{\refdefined{com.retroMachines.ui.screens.menus.CreateProfileMenuScreen}}, LoadMenuScreen\small{\refdefined{com.retroMachines.ui.screens.menus.LoadMenuScreen}}, LevelMenuScreen\small{\refdefined{com.retroMachines.ui.screens.menus.LevelMenuScreen}}, MenuScreen\small{\refdefined{com.retroMachines.ui.screens.menus.MenuScreen}}, MainMenuScreen\small{\refdefined{com.retroMachines.ui.screens.menus.MainMenuScreen}}, ProfileSettingsMenuScreen\small{\refdefined{com.retroMachines.ui.screens.menus.ProfileSettingsMenuScreen}}, EvaluationScreen\small{\refdefined{com.retroMachines.ui.screens.game.EvaluationScreen}}, GameScreen\small{\refdefined{com.retroMachines.ui.screens.game.GameScreen}}}
\subsection{Constructor summary}{
\begin{verse}
{\bf AbstractScreen(RetroMachines)} \\
\end{verse}
}
\subsection{Method summary}{
\begin{verse}
{\bf dispose()} \\
{\bf hide()} \\
{\bf pause()} \\
{\bf render(float)} Renders the Stage to the Screen.\\
{\bf resize(int, int)} \\
{\bf resume()} \\
{\bf show()} \\
\end{verse}
}
\subsection{Constructors}{
\vskip -2em
\begin{itemize}
\item{ 
\index{AbstractScreen(RetroMachines)}
{\bf  AbstractScreen}\\
\texttt{public\ {\bf  AbstractScreen}(\texttt{com.retroMachines.RetroMachines} {\bf  game})
\label{com.retroMachines.ui.screens.AbstractScreen(com.retroMachines.RetroMachines)}}%end signature
}%end item
\end{itemize}
}
\subsection{Methods}{
\vskip -2em
\begin{itemize}
\item{ 
\index{dispose()}
{\bf  dispose}\\
\texttt{ void\ {\bf  dispose}()
\label{com.retroMachines.ui.screens.AbstractScreen.dispose()}}%end signature
}%end item
\item{ 
\index{hide()}
{\bf  hide}\\
\texttt{ void\ {\bf  hide}()
\label{com.retroMachines.ui.screens.AbstractScreen.hide()}}%end signature
}%end item
\item{ 
\index{pause()}
{\bf  pause}\\
\texttt{ void\ {\bf  pause}()
\label{com.retroMachines.ui.screens.AbstractScreen.pause()}}%end signature
}%end item
\item{ 
\index{render(float)}
{\bf  render}\\
\texttt{public void\ {\bf  render}(\texttt{float} {\bf  delta})
\label{com.retroMachines.ui.screens.AbstractScreen.render(float)}}%end signature
\begin{itemize}
\item{
{\bf  Description}

Renders the Stage to the Screen.
}
\end{itemize}
}%end item
\item{ 
\index{resize(int, int)}
{\bf  resize}\\
\texttt{ void\ {\bf  resize}(\texttt{int} {\bf  arg0},
\texttt{int} {\bf  arg1})
\label{com.retroMachines.ui.screens.AbstractScreen.resize(int, int)}}%end signature
}%end item
\item{ 
\index{resume()}
{\bf  resume}\\
\texttt{ void\ {\bf  resume}()
\label{com.retroMachines.ui.screens.AbstractScreen.resume()}}%end signature
}%end item
\item{ 
\index{show()}
{\bf  show}\\
\texttt{ void\ {\bf  show}()
\label{com.retroMachines.ui.screens.AbstractScreen.show()}}%end signature
}%end item
\end{itemize}
}
}
}
\chapter{Package com.retroMachines}{
\label{com.retroMachines}\hskip -.05in
\hbox to \hsize{\textit{ Package Contents\hfil Page}}
\vskip .13in
\hbox{{\bf  Classes}}
\entityintro{RetroMachines}{com.retroMachines.RetroMachines}{}
\vskip .1in
\vskip .1in
\section{\label{com.retroMachines.RetroMachines}\index{RetroMachines}Class RetroMachines}{
\vskip .1in 
\subsection{Declaration}{
\small public class RetroMachines
\\ {\bf  extends} com.badlogic.gdx.Game
\refdefined{com.badlogic.gdx.Game}}
\subsection{Field summary}{
\begin{verse}
{\bf HEIGHT} \\
{\bf WIDTH} \\
\end{verse}
}
\subsection{Constructor summary}{
\begin{verse}
{\bf RetroMachines()} \\
\end{verse}
}
\subsection{Method summary}{
\begin{verse}
{\bf create()} initializes the game (all the controllers) after started by the Android Launcher.\\
{\bf getProfileController()} \\
{\bf getSettingController()} \\
\end{verse}
}
\subsection{Fields}{
\begin{itemize}
\item{
\index{WIDTH}
\label{com.retroMachines.RetroMachines.WIDTH}public static final int {\bf  WIDTH}}
\item{
\index{HEIGHT}
\label{com.retroMachines.RetroMachines.HEIGHT}public static final int {\bf  HEIGHT}}
\end{itemize}
}
\subsection{Constructors}{
\vskip -2em
\begin{itemize}
\item{ 
\index{RetroMachines()}
{\bf  RetroMachines}\\
\texttt{public\ {\bf  RetroMachines}()
\label{com.retroMachines.RetroMachines()}}%end signature
}%end item
\end{itemize}
}
\subsection{Methods}{
\vskip -2em
\begin{itemize}
\item{ 
\index{create()}
{\bf  create}\\
\texttt{public void\ {\bf  create}()
\label{com.retroMachines.RetroMachines.create()}}%end signature
\begin{itemize}
\item{
{\bf  Description}

initializes the game (all the controllers) after started by the Android Launcher. Afterwards it displays the loading screen to the user.
}
\end{itemize}
}%end item
\item{ 
\index{getProfileController()}
{\bf  getProfileController}\\
\texttt{public game.controllers.ProfileController\ {\bf  getProfileController}()
\label{com.retroMachines.RetroMachines.getProfileController()}}%end signature
}%end item
\item{ 
\index{getSettingController()}
{\bf  getSettingController}\\
\texttt{public game.controllers.SettingController\ {\bf  getSettingController}()
\label{com.retroMachines.RetroMachines.getSettingController()}}%end signature
}%end item
\end{itemize}
}
\subsection{Members inherited from class Game }{
\texttt{com.badlogic.gdx.Game} {\small 
\refdefined{com.badlogic.gdx.Game}}
{\small 

\vskip -2em
\begin{itemize}
\item{\vskip -1.5ex 
\texttt{public void {\bf  dispose}()
}%end signature
}%end item
\item{\vskip -1.5ex 
\texttt{public Screen {\bf  getScreen}()
}%end signature
}%end item
\item{\vskip -1.5ex 
\texttt{public void {\bf  pause}()
}%end signature
}%end item
\item{\vskip -1.5ex 
\texttt{public void {\bf  render}()
}%end signature
}%end item
\item{\vskip -1.5ex 
\texttt{public void {\bf  resize}(\texttt{int} {\bf  arg0},
\texttt{int} {\bf  arg1})
}%end signature
}%end item
\item{\vskip -1.5ex 
\texttt{public void {\bf  resume}()
}%end signature
}%end item
\item{\vskip -1.5ex 
\texttt{public void {\bf  setScreen}(\texttt{Screen} {\bf  arg0})
}%end signature
}%end item
\end{itemize}
}
}
}
\chapter{Package com.retroMachines.ui.screens.menus}{
\label{com.retroMachines.ui.screens.menus}\hskip -.05in
\hbox to \hsize{\textit{ Package Contents\hfil Page}}
\vskip .13in
\hbox{{\bf  Classes}}
\entityintro{AboutMenuScreen}{com.retroMachines.ui.screens.menus.AboutMenuScreen}{}
\entityintro{CreateProfileMenuScreen}{com.retroMachines.ui.screens.menus.CreateProfileMenuScreen}{}
\entityintro{LevelMenuScreen}{com.retroMachines.ui.screens.menus.LevelMenuScreen}{}
\entityintro{LoadMenuScreen}{com.retroMachines.ui.screens.menus.LoadMenuScreen}{loading screen which appears after starting the App}
\entityintro{MainMenuScreen}{com.retroMachines.ui.screens.menus.MainMenuScreen}{}
\entityintro{MenuScreen}{com.retroMachines.ui.screens.menus.MenuScreen}{abstrakte MenuScreen klasse die die grundsätzliche Struktur eines Menüs wiederspiegelt.}
\entityintro{ProfileMenuScreen}{com.retroMachines.ui.screens.menus.ProfileMenuScreen}{}
\entityintro{ProfileSettingsMenuScreen}{com.retroMachines.ui.screens.menus.ProfileSettingsMenuScreen}{}
\entityintro{SettingsMenuScreen}{com.retroMachines.ui.screens.menus.SettingsMenuScreen}{}
\vskip .1in
\vskip .1in
\section{\label{com.retroMachines.ui.screens.menus.AboutMenuScreen}\index{AboutMenuScreen}Class AboutMenuScreen}{
\vskip .1in 
\subsection{Declaration}{
\small public class AboutMenuScreen
\\ {\bf  extends} com.retroMachines.ui.screens.AbstractScreen
\refdefined{com.retroMachines.ui.screens.AbstractScreen}}
\subsection{Constructor summary}{
\begin{verse}
{\bf AboutMenuScreen(RetroMachines)} \\
\end{verse}
}
\subsection{Method summary}{
\begin{verse}
{\bf show()} \\
\end{verse}
}
\subsection{Constructors}{
\vskip -2em
\begin{itemize}
\item{ 
\index{AboutMenuScreen(RetroMachines)}
{\bf  AboutMenuScreen}\\
\texttt{public\ {\bf  AboutMenuScreen}(\texttt{com.retroMachines.RetroMachines} {\bf  game})
\label{com.retroMachines.ui.screens.menus.AboutMenuScreen(com.retroMachines.RetroMachines)}}%end signature
}%end item
\end{itemize}
}
\subsection{Methods}{
\vskip -2em
\begin{itemize}
\item{ 
\index{show()}
{\bf  show}\\
\texttt{ void\ {\bf  show}()
\label{com.retroMachines.ui.screens.menus.AboutMenuScreen.show()}}%end signature
}%end item
\end{itemize}
}
\subsection{Members inherited from class AbstractScreen }{
\texttt{com.retroMachines.ui.screens.AbstractScreen} {\small 
\refdefined{com.retroMachines.ui.screens.AbstractScreen}}
{\small 

\vskip -2em
\begin{itemize}
\item{\vskip -1.5ex 
\texttt{public void {\bf  dispose}()
}%end signature
}%end item
\item{\vskip -1.5ex 
\texttt{public void {\bf  hide}()
}%end signature
}%end item
\item{\vskip -1.5ex 
\texttt{public void {\bf  pause}()
}%end signature
}%end item
\item{\vskip -1.5ex 
\texttt{public void {\bf  render}(\texttt{float} {\bf  delta})
}%end signature
}%end item
\item{\vskip -1.5ex 
\texttt{public void {\bf  resize}(\texttt{int} {\bf  width},
\texttt{int} {\bf  height})
}%end signature
}%end item
\item{\vskip -1.5ex 
\texttt{public void {\bf  resume}()
}%end signature
}%end item
\item{\vskip -1.5ex 
\texttt{public void {\bf  show}()
}%end signature
}%end item
\end{itemize}
}
}
\section{\label{com.retroMachines.ui.screens.menus.CreateProfileMenuScreen}\index{CreateProfileMenuScreen}Class CreateProfileMenuScreen}{
\vskip .1in 
\subsection{Declaration}{
\small public class CreateProfileMenuScreen
\\ {\bf  extends} com.retroMachines.ui.screens.menus.MenuScreen
\refdefined{com.retroMachines.ui.screens.menus.MenuScreen}}
\subsection{Constructor summary}{
\begin{verse}
{\bf CreateProfileMenuScreen(RetroMachines)} \\
\end{verse}
}
\subsection{Constructors}{
\vskip -2em
\begin{itemize}
\item{ 
\index{CreateProfileMenuScreen(RetroMachines)}
{\bf  CreateProfileMenuScreen}\\
\texttt{public\ {\bf  CreateProfileMenuScreen}(\texttt{com.retroMachines.RetroMachines} {\bf  game})
\label{com.retroMachines.ui.screens.menus.CreateProfileMenuScreen(com.retroMachines.RetroMachines)}}%end signature
}%end item
\end{itemize}
}
\subsection{Members inherited from class AbstractScreen }{
\texttt{com.retroMachines.ui.screens.AbstractScreen} {\small 
\refdefined{com.retroMachines.ui.screens.AbstractScreen}}
{\small 

\vskip -2em
\begin{itemize}
\item{\vskip -1.5ex 
\texttt{public void {\bf  dispose}()
}%end signature
}%end item
\item{\vskip -1.5ex 
\texttt{public void {\bf  hide}()
}%end signature
}%end item
\item{\vskip -1.5ex 
\texttt{public void {\bf  pause}()
}%end signature
}%end item
\item{\vskip -1.5ex 
\texttt{public void {\bf  render}(\texttt{float} {\bf  delta})
}%end signature
}%end item
\item{\vskip -1.5ex 
\texttt{public void {\bf  resize}(\texttt{int} {\bf  width},
\texttt{int} {\bf  height})
}%end signature
}%end item
\item{\vskip -1.5ex 
\texttt{public void {\bf  resume}()
}%end signature
}%end item
\item{\vskip -1.5ex 
\texttt{public void {\bf  show}()
}%end signature
}%end item
\end{itemize}
}
}
\section{\label{com.retroMachines.ui.screens.menus.LevelMenuScreen}\index{LevelMenuScreen}Class LevelMenuScreen}{
\vskip .1in 
\subsection{Declaration}{
\small public class LevelMenuScreen
\\ {\bf  extends} com.retroMachines.ui.screens.menus.MenuScreen
\refdefined{com.retroMachines.ui.screens.menus.MenuScreen}}
\subsection{Constructor summary}{
\begin{verse}
{\bf LevelMenuScreen(RetroMachines)} \\
\end{verse}
}
\subsection{Constructors}{
\vskip -2em
\begin{itemize}
\item{ 
\index{LevelMenuScreen(RetroMachines)}
{\bf  LevelMenuScreen}\\
\texttt{public\ {\bf  LevelMenuScreen}(\texttt{com.retroMachines.RetroMachines} {\bf  game})
\label{com.retroMachines.ui.screens.menus.LevelMenuScreen(com.retroMachines.RetroMachines)}}%end signature
}%end item
\end{itemize}
}
\subsection{Members inherited from class AbstractScreen }{
\texttt{com.retroMachines.ui.screens.AbstractScreen} {\small 
\refdefined{com.retroMachines.ui.screens.AbstractScreen}}
{\small 

\vskip -2em
\begin{itemize}
\item{\vskip -1.5ex 
\texttt{public void {\bf  dispose}()
}%end signature
}%end item
\item{\vskip -1.5ex 
\texttt{public void {\bf  hide}()
}%end signature
}%end item
\item{\vskip -1.5ex 
\texttt{public void {\bf  pause}()
}%end signature
}%end item
\item{\vskip -1.5ex 
\texttt{public void {\bf  render}(\texttt{float} {\bf  delta})
}%end signature
}%end item
\item{\vskip -1.5ex 
\texttt{public void {\bf  resize}(\texttt{int} {\bf  width},
\texttt{int} {\bf  height})
}%end signature
}%end item
\item{\vskip -1.5ex 
\texttt{public void {\bf  resume}()
}%end signature
}%end item
\item{\vskip -1.5ex 
\texttt{public void {\bf  show}()
}%end signature
}%end item
\end{itemize}
}
}
\section{\label{com.retroMachines.ui.screens.menus.LoadMenuScreen}\index{LoadMenuScreen}Class LoadMenuScreen}{
\vskip .1in 
loading screen which appears after starting the App\vskip .1in 
\subsection{Declaration}{
\small public class LoadMenuScreen
\\ {\bf  extends} com.retroMachines.ui.screens.AbstractScreen
\refdefined{com.retroMachines.ui.screens.AbstractScreen}}
\subsection{Constructor summary}{
\begin{verse}
{\bf LoadMenuScreen(RetroMachines)} \\
\end{verse}
}
\subsection{Method summary}{
\begin{verse}
{\bf show()} the screen is displayed\\
\end{verse}
}
\subsection{Constructors}{
\vskip -2em
\begin{itemize}
\item{ 
\index{LoadMenuScreen(RetroMachines)}
{\bf  LoadMenuScreen}\\
\texttt{public\ {\bf  LoadMenuScreen}(\texttt{com.retroMachines.RetroMachines} {\bf  game})
\label{com.retroMachines.ui.screens.menus.LoadMenuScreen(com.retroMachines.RetroMachines)}}%end signature
\begin{itemize}
\item{
{\bf  Parameters}
  \begin{itemize}
   \item{
\texttt{game} -- the game which should be loaded while the screen is displayed}
  \end{itemize}
}%end item
\end{itemize}
}%end item
\end{itemize}
}
\subsection{Methods}{
\vskip -2em
\begin{itemize}
\item{ 
\index{show()}
{\bf  show}\\
\texttt{public void\ {\bf  show}()
\label{com.retroMachines.ui.screens.menus.LoadMenuScreen.show()}}%end signature
\begin{itemize}
\item{
{\bf  Description}

the screen is displayed
}
\end{itemize}
}%end item
\end{itemize}
}
\subsection{Members inherited from class AbstractScreen }{
\texttt{com.retroMachines.ui.screens.AbstractScreen} {\small 
\refdefined{com.retroMachines.ui.screens.AbstractScreen}}
{\small 

\vskip -2em
\begin{itemize}
\item{\vskip -1.5ex 
\texttt{public void {\bf  dispose}()
}%end signature
}%end item
\item{\vskip -1.5ex 
\texttt{public void {\bf  hide}()
}%end signature
}%end item
\item{\vskip -1.5ex 
\texttt{public void {\bf  pause}()
}%end signature
}%end item
\item{\vskip -1.5ex 
\texttt{public void {\bf  render}(\texttt{float} {\bf  delta})
}%end signature
}%end item
\item{\vskip -1.5ex 
\texttt{public void {\bf  resize}(\texttt{int} {\bf  width},
\texttt{int} {\bf  height})
}%end signature
}%end item
\item{\vskip -1.5ex 
\texttt{public void {\bf  resume}()
}%end signature
}%end item
\item{\vskip -1.5ex 
\texttt{public void {\bf  show}()
}%end signature
}%end item
\end{itemize}
}
}
\section{\label{com.retroMachines.ui.screens.menus.MainMenuScreen}\index{MainMenuScreen}Class MainMenuScreen}{
\vskip .1in 
\subsection{Declaration}{
\small public class MainMenuScreen
\\ {\bf  extends} com.retroMachines.ui.screens.menus.MenuScreen
\refdefined{com.retroMachines.ui.screens.menus.MenuScreen}}
\subsection{Constructor summary}{
\begin{verse}
{\bf MainMenuScreen(RetroMachines)} \\
\end{verse}
}
\subsection{Method summary}{
\begin{verse}
{\bf show()} \\
\end{verse}
}
\subsection{Constructors}{
\vskip -2em
\begin{itemize}
\item{ 
\index{MainMenuScreen(RetroMachines)}
{\bf  MainMenuScreen}\\
\texttt{public\ {\bf  MainMenuScreen}(\texttt{com.retroMachines.RetroMachines} {\bf  game})
\label{com.retroMachines.ui.screens.menus.MainMenuScreen(com.retroMachines.RetroMachines)}}%end signature
}%end item
\end{itemize}
}
\subsection{Methods}{
\vskip -2em
\begin{itemize}
\item{ 
\index{show()}
{\bf  show}\\
\texttt{ void\ {\bf  show}()
\label{com.retroMachines.ui.screens.menus.MainMenuScreen.show()}}%end signature
}%end item
\end{itemize}
}
\subsection{Members inherited from class AbstractScreen }{
\texttt{com.retroMachines.ui.screens.AbstractScreen} {\small 
\refdefined{com.retroMachines.ui.screens.AbstractScreen}}
{\small 

\vskip -2em
\begin{itemize}
\item{\vskip -1.5ex 
\texttt{public void {\bf  dispose}()
}%end signature
}%end item
\item{\vskip -1.5ex 
\texttt{public void {\bf  hide}()
}%end signature
}%end item
\item{\vskip -1.5ex 
\texttt{public void {\bf  pause}()
}%end signature
}%end item
\item{\vskip -1.5ex 
\texttt{public void {\bf  render}(\texttt{float} {\bf  delta})
}%end signature
}%end item
\item{\vskip -1.5ex 
\texttt{public void {\bf  resize}(\texttt{int} {\bf  width},
\texttt{int} {\bf  height})
}%end signature
}%end item
\item{\vskip -1.5ex 
\texttt{public void {\bf  resume}()
}%end signature
}%end item
\item{\vskip -1.5ex 
\texttt{public void {\bf  show}()
}%end signature
}%end item
\end{itemize}
}
}
\section{\label{com.retroMachines.ui.screens.menus.MenuScreen}\index{MenuScreen}Class MenuScreen}{
\vskip .1in 
abstrakte MenuScreen klasse die die grundsätzliche Struktur eines Menüs wiederspiegelt.\vskip .1in 
\subsection{Declaration}{
\small public abstract class MenuScreen
\\ {\bf  extends} com.retroMachines.ui.screens.AbstractScreen
\refdefined{com.retroMachines.ui.screens.AbstractScreen}}
\subsection{All known subclasses}{ProfileMenuScreen\small{\refdefined{com.retroMachines.ui.screens.menus.ProfileMenuScreen}}, SettingsMenuScreen\small{\refdefined{com.retroMachines.ui.screens.menus.SettingsMenuScreen}}, CreateProfileMenuScreen\small{\refdefined{com.retroMachines.ui.screens.menus.CreateProfileMenuScreen}}, LevelMenuScreen\small{\refdefined{com.retroMachines.ui.screens.menus.LevelMenuScreen}}, MainMenuScreen\small{\refdefined{com.retroMachines.ui.screens.menus.MainMenuScreen}}, ProfileSettingsMenuScreen\small{\refdefined{com.retroMachines.ui.screens.menus.ProfileSettingsMenuScreen}}}
\subsection{Constructor summary}{
\begin{verse}
{\bf MenuScreen(RetroMachines)} \\
\end{verse}
}
\subsection{Constructors}{
\vskip -2em
\begin{itemize}
\item{ 
\index{MenuScreen(RetroMachines)}
{\bf  MenuScreen}\\
\texttt{public\ {\bf  MenuScreen}(\texttt{com.retroMachines.RetroMachines} {\bf  game})
\label{com.retroMachines.ui.screens.menus.MenuScreen(com.retroMachines.RetroMachines)}}%end signature
}%end item
\end{itemize}
}
\subsection{Members inherited from class AbstractScreen }{
\texttt{com.retroMachines.ui.screens.AbstractScreen} {\small 
\refdefined{com.retroMachines.ui.screens.AbstractScreen}}
{\small 

\vskip -2em
\begin{itemize}
\item{\vskip -1.5ex 
\texttt{public void {\bf  dispose}()
}%end signature
}%end item
\item{\vskip -1.5ex 
\texttt{public void {\bf  hide}()
}%end signature
}%end item
\item{\vskip -1.5ex 
\texttt{public void {\bf  pause}()
}%end signature
}%end item
\item{\vskip -1.5ex 
\texttt{public void {\bf  render}(\texttt{float} {\bf  delta})
}%end signature
}%end item
\item{\vskip -1.5ex 
\texttt{public void {\bf  resize}(\texttt{int} {\bf  width},
\texttt{int} {\bf  height})
}%end signature
}%end item
\item{\vskip -1.5ex 
\texttt{public void {\bf  resume}()
}%end signature
}%end item
\item{\vskip -1.5ex 
\texttt{public void {\bf  show}()
}%end signature
}%end item
\end{itemize}
}
}
\section{\label{com.retroMachines.ui.screens.menus.ProfileMenuScreen}\index{ProfileMenuScreen}Class ProfileMenuScreen}{
\vskip .1in 
\subsection{Declaration}{
\small public class ProfileMenuScreen
\\ {\bf  extends} com.retroMachines.ui.screens.menus.MenuScreen
\refdefined{com.retroMachines.ui.screens.menus.MenuScreen}}
\subsection{Constructor summary}{
\begin{verse}
{\bf ProfileMenuScreen(RetroMachines)} \\
\end{verse}
}
\subsection{Constructors}{
\vskip -2em
\begin{itemize}
\item{ 
\index{ProfileMenuScreen(RetroMachines)}
{\bf  ProfileMenuScreen}\\
\texttt{public\ {\bf  ProfileMenuScreen}(\texttt{com.retroMachines.RetroMachines} {\bf  game})
\label{com.retroMachines.ui.screens.menus.ProfileMenuScreen(com.retroMachines.RetroMachines)}}%end signature
}%end item
\end{itemize}
}
\subsection{Members inherited from class AbstractScreen }{
\texttt{com.retroMachines.ui.screens.AbstractScreen} {\small 
\refdefined{com.retroMachines.ui.screens.AbstractScreen}}
{\small 

\vskip -2em
\begin{itemize}
\item{\vskip -1.5ex 
\texttt{public void {\bf  dispose}()
}%end signature
}%end item
\item{\vskip -1.5ex 
\texttt{public void {\bf  hide}()
}%end signature
}%end item
\item{\vskip -1.5ex 
\texttt{public void {\bf  pause}()
}%end signature
}%end item
\item{\vskip -1.5ex 
\texttt{public void {\bf  render}(\texttt{float} {\bf  delta})
}%end signature
}%end item
\item{\vskip -1.5ex 
\texttt{public void {\bf  resize}(\texttt{int} {\bf  width},
\texttt{int} {\bf  height})
}%end signature
}%end item
\item{\vskip -1.5ex 
\texttt{public void {\bf  resume}()
}%end signature
}%end item
\item{\vskip -1.5ex 
\texttt{public void {\bf  show}()
}%end signature
}%end item
\end{itemize}
}
}
\section{\label{com.retroMachines.ui.screens.menus.ProfileSettingsMenuScreen}\index{ProfileSettingsMenuScreen}Class ProfileSettingsMenuScreen}{
\vskip .1in 
\subsection{Declaration}{
\small public class ProfileSettingsMenuScreen
\\ {\bf  extends} com.retroMachines.ui.screens.menus.MenuScreen
\refdefined{com.retroMachines.ui.screens.menus.MenuScreen}}
\subsection{Constructor summary}{
\begin{verse}
{\bf ProfileSettingsMenuScreen(RetroMachines)} \\
\end{verse}
}
\subsection{Constructors}{
\vskip -2em
\begin{itemize}
\item{ 
\index{ProfileSettingsMenuScreen(RetroMachines)}
{\bf  ProfileSettingsMenuScreen}\\
\texttt{public\ {\bf  ProfileSettingsMenuScreen}(\texttt{com.retroMachines.RetroMachines} {\bf  game})
\label{com.retroMachines.ui.screens.menus.ProfileSettingsMenuScreen(com.retroMachines.RetroMachines)}}%end signature
}%end item
\end{itemize}
}
\subsection{Members inherited from class AbstractScreen }{
\texttt{com.retroMachines.ui.screens.AbstractScreen} {\small 
\refdefined{com.retroMachines.ui.screens.AbstractScreen}}
{\small 

\vskip -2em
\begin{itemize}
\item{\vskip -1.5ex 
\texttt{public void {\bf  dispose}()
}%end signature
}%end item
\item{\vskip -1.5ex 
\texttt{public void {\bf  hide}()
}%end signature
}%end item
\item{\vskip -1.5ex 
\texttt{public void {\bf  pause}()
}%end signature
}%end item
\item{\vskip -1.5ex 
\texttt{public void {\bf  render}(\texttt{float} {\bf  delta})
}%end signature
}%end item
\item{\vskip -1.5ex 
\texttt{public void {\bf  resize}(\texttt{int} {\bf  width},
\texttt{int} {\bf  height})
}%end signature
}%end item
\item{\vskip -1.5ex 
\texttt{public void {\bf  resume}()
}%end signature
}%end item
\item{\vskip -1.5ex 
\texttt{public void {\bf  show}()
}%end signature
}%end item
\end{itemize}
}
}
\section{\label{com.retroMachines.ui.screens.menus.SettingsMenuScreen}\index{SettingsMenuScreen}Class SettingsMenuScreen}{
\vskip .1in 
\subsection{Declaration}{
\small public class SettingsMenuScreen
\\ {\bf  extends} com.retroMachines.ui.screens.menus.MenuScreen
\refdefined{com.retroMachines.ui.screens.menus.MenuScreen}}
\subsection{Constructor summary}{
\begin{verse}
{\bf SettingsMenuScreen(RetroMachines)} \\
\end{verse}
}
\subsection{Constructors}{
\vskip -2em
\begin{itemize}
\item{ 
\index{SettingsMenuScreen(RetroMachines)}
{\bf  SettingsMenuScreen}\\
\texttt{public\ {\bf  SettingsMenuScreen}(\texttt{com.retroMachines.RetroMachines} {\bf  game})
\label{com.retroMachines.ui.screens.menus.SettingsMenuScreen(com.retroMachines.RetroMachines)}}%end signature
}%end item
\end{itemize}
}
\subsection{Members inherited from class AbstractScreen }{
\texttt{com.retroMachines.ui.screens.AbstractScreen} {\small 
\refdefined{com.retroMachines.ui.screens.AbstractScreen}}
{\small 

\vskip -2em
\begin{itemize}
\item{\vskip -1.5ex 
\texttt{public void {\bf  dispose}()
}%end signature
}%end item
\item{\vskip -1.5ex 
\texttt{public void {\bf  hide}()
}%end signature
}%end item
\item{\vskip -1.5ex 
\texttt{public void {\bf  pause}()
}%end signature
}%end item
\item{\vskip -1.5ex 
\texttt{public void {\bf  render}(\texttt{float} {\bf  delta})
}%end signature
}%end item
\item{\vskip -1.5ex 
\texttt{public void {\bf  resize}(\texttt{int} {\bf  width},
\texttt{int} {\bf  height})
}%end signature
}%end item
\item{\vskip -1.5ex 
\texttt{public void {\bf  resume}()
}%end signature
}%end item
\item{\vskip -1.5ex 
\texttt{public void {\bf  show}()
}%end signature
}%end item
\end{itemize}
}
}
}
\chapter{Package com.retroMachines.ui.screens.game}{
\label{com.retroMachines.ui.screens.game}\hskip -.05in
\hbox to \hsize{\textit{ Package Contents\hfil Page}}
\vskip .13in
\hbox{{\bf  Classes}}
\entityintro{EvaluationScreen}{com.retroMachines.ui.screens.game.EvaluationScreen}{}
\entityintro{GameScreen}{com.retroMachines.ui.screens.game.GameScreen}{}
\vskip .1in
\vskip .1in
\section{\label{com.retroMachines.ui.screens.game.EvaluationScreen}\index{EvaluationScreen}Class EvaluationScreen}{
\vskip .1in 
\subsection{Declaration}{
\small public class EvaluationScreen
\\ {\bf  extends} com.retroMachines.ui.screens.AbstractScreen
\refdefined{com.retroMachines.ui.screens.AbstractScreen}\\ {\bf  implements} 
com.badlogic.gdx.physics.box2d.ContactListener}
\subsection{Constructor summary}{
\begin{verse}
{\bf EvaluationScreen(RetroMachines, GameController)} \\
\end{verse}
}
\subsection{Method summary}{
\begin{verse}
{\bf beginContact(Contact)} ContactListener Section\\
{\bf endContact(Contact)} \\
{\bf pauseAnimation()} sets animationInProgress to false and freezes the animation in it's current position\\
{\bf postSolve(Contact, ContactImpulse)} \\
{\bf preSolve(Contact, Manifold)} \\
{\bf render(float)} \\
{\bf setLambaTerm(Tree)} assigns a lambda term to the screen for the animation\\
{\bf startAnimation()} sets animationProgress to true and triggers the animation and displays it to the user\\
\end{verse}
}
\subsection{Constructors}{
\vskip -2em
\begin{itemize}
\item{ 
\index{EvaluationScreen(RetroMachines, GameController)}
{\bf  EvaluationScreen}\\
\texttt{public\ {\bf  EvaluationScreen}(\texttt{com.retroMachines.RetroMachines} {\bf  game},
\texttt{com.retroMachines.game.controllers.GameController} {\bf  gameController})
\label{com.retroMachines.ui.screens.game.EvaluationScreen(com.retroMachines.RetroMachines, com.retroMachines.game.controllers.GameController)}}%end signature
}%end item
\end{itemize}
}
\subsection{Methods}{
\vskip -2em
\begin{itemize}
\item{ 
\index{beginContact(Contact)}
{\bf  beginContact}\\
\texttt{public void\ {\bf  beginContact}(\texttt{com.badlogic.gdx.physics.box2d.Contact} {\bf  contact})
\label{com.retroMachines.ui.screens.game.EvaluationScreen.beginContact(com.badlogic.gdx.physics.box2d.Contact)}}%end signature
\begin{itemize}
\item{
{\bf  Description}

ContactListener Section
}
\end{itemize}
}%end item
\item{ 
\index{endContact(Contact)}
{\bf  endContact}\\
\texttt{ void\ {\bf  endContact}(\texttt{com.badlogic.gdx.physics.box2d.Contact} {\bf  arg0})
\label{com.retroMachines.ui.screens.game.EvaluationScreen.endContact(com.badlogic.gdx.physics.box2d.Contact)}}%end signature
}%end item
\item{ 
\index{pauseAnimation()}
{\bf  pauseAnimation}\\
\texttt{public void\ {\bf  pauseAnimation}()
\label{com.retroMachines.ui.screens.game.EvaluationScreen.pauseAnimation()}}%end signature
\begin{itemize}
\item{
{\bf  Description}

sets animationInProgress to false and freezes the animation in it's current position
}
\end{itemize}
}%end item
\item{ 
\index{postSolve(Contact, ContactImpulse)}
{\bf  postSolve}\\
\texttt{ void\ {\bf  postSolve}(\texttt{com.badlogic.gdx.physics.box2d.Contact} {\bf  arg0},
\texttt{com.badlogic.gdx.physics.box2d.ContactImpulse} {\bf  arg1})
\label{com.retroMachines.ui.screens.game.EvaluationScreen.postSolve(com.badlogic.gdx.physics.box2d.Contact, com.badlogic.gdx.physics.box2d.ContactImpulse)}}%end signature
}%end item
\item{ 
\index{preSolve(Contact, Manifold)}
{\bf  preSolve}\\
\texttt{ void\ {\bf  preSolve}(\texttt{com.badlogic.gdx.physics.box2d.Contact} {\bf  arg0},
\texttt{com.badlogic.gdx.physics.box2d.Manifold} {\bf  arg1})
\label{com.retroMachines.ui.screens.game.EvaluationScreen.preSolve(com.badlogic.gdx.physics.box2d.Contact, com.badlogic.gdx.physics.box2d.Manifold)}}%end signature
}%end item
\item{ 
\index{render(float)}
{\bf  render}\\
\texttt{public void\ {\bf  render}(\texttt{float} {\bf  delta})
\label{com.retroMachines.ui.screens.game.EvaluationScreen.render(float)}}%end signature
\begin{itemize}
\item{
{\bf  Description copied from com.retroMachines.ui.screens.AbstractScreen{\small \refdefined{com.retroMachines.ui.screens.AbstractScreen}} }

Renders the Stage to the Screen.
}
\end{itemize}
}%end item
\item{ 
\index{setLambaTerm(Tree)}
{\bf  setLambaTerm}\\
\texttt{public void\ {\bf  setLambaTerm}(\texttt{com.retroMachines.util.lambda.Tree} {\bf  t})
\label{com.retroMachines.ui.screens.game.EvaluationScreen.setLambaTerm(com.retroMachines.util.lambda.Tree)}}%end signature
\begin{itemize}
\item{
{\bf  Description}

assigns a lambda term to the screen for the animation
}
\item{
{\bf  Parameters}
  \begin{itemize}
   \item{
\texttt{t} -- the lambda term in question}
  \end{itemize}
}%end item
\end{itemize}
}%end item
\item{ 
\index{startAnimation()}
{\bf  startAnimation}\\
\texttt{public void\ {\bf  startAnimation}()
\label{com.retroMachines.ui.screens.game.EvaluationScreen.startAnimation()}}%end signature
\begin{itemize}
\item{
{\bf  Description}

sets animationProgress to true and triggers the animation and displays it to the user
}
\end{itemize}
}%end item
\end{itemize}
}
\subsection{Members inherited from class AbstractScreen }{
\texttt{com.retroMachines.ui.screens.AbstractScreen} {\small 
\refdefined{com.retroMachines.ui.screens.AbstractScreen}}
{\small 

\vskip -2em
\begin{itemize}
\item{\vskip -1.5ex 
\texttt{public void {\bf  dispose}()
}%end signature
}%end item
\item{\vskip -1.5ex 
\texttt{public void {\bf  hide}()
}%end signature
}%end item
\item{\vskip -1.5ex 
\texttt{public void {\bf  pause}()
}%end signature
}%end item
\item{\vskip -1.5ex 
\texttt{public void {\bf  render}(\texttt{float} {\bf  delta})
}%end signature
}%end item
\item{\vskip -1.5ex 
\texttt{public void {\bf  resize}(\texttt{int} {\bf  width},
\texttt{int} {\bf  height})
}%end signature
}%end item
\item{\vskip -1.5ex 
\texttt{public void {\bf  resume}()
}%end signature
}%end item
\item{\vskip -1.5ex 
\texttt{public void {\bf  show}()
}%end signature
}%end item
\end{itemize}
}
}
\section{\label{com.retroMachines.ui.screens.game.GameScreen}\index{GameScreen}Class GameScreen}{
\vskip .1in 
\subsection{Declaration}{
\small public class GameScreen
\\ {\bf  extends} com.retroMachines.ui.screens.AbstractScreen
\refdefined{com.retroMachines.ui.screens.AbstractScreen}}
\subsection{Constructor summary}{
\begin{verse}
{\bf GameScreen(RetroMachines, GameController)} \\
\end{verse}
}
\subsection{Method summary}{
\begin{verse}
{\bf dispose()} \\
{\bf render(float)} \\
{\bf setMap(TiledMap)} \\
\end{verse}
}
\subsection{Constructors}{
\vskip -2em
\begin{itemize}
\item{ 
\index{GameScreen(RetroMachines, GameController)}
{\bf  GameScreen}\\
\texttt{public\ {\bf  GameScreen}(\texttt{com.retroMachines.RetroMachines} {\bf  game},
\texttt{com.retroMachines.game.controllers.GameController} {\bf  gameController})
\label{com.retroMachines.ui.screens.game.GameScreen(com.retroMachines.RetroMachines, com.retroMachines.game.controllers.GameController)}}%end signature
}%end item
\end{itemize}
}
\subsection{Methods}{
\vskip -2em
\begin{itemize}
\item{ 
\index{dispose()}
{\bf  dispose}\\
\texttt{ void\ {\bf  dispose}()
\label{com.retroMachines.ui.screens.game.GameScreen.dispose()}}%end signature
}%end item
\item{ 
\index{render(float)}
{\bf  render}\\
\texttt{public void\ {\bf  render}(\texttt{float} {\bf  delta})
\label{com.retroMachines.ui.screens.game.GameScreen.render(float)}}%end signature
\begin{itemize}
\item{
{\bf  Description copied from com.retroMachines.ui.screens.AbstractScreen{\small \refdefined{com.retroMachines.ui.screens.AbstractScreen}} }

Renders the Stage to the Screen.
}
\end{itemize}
}%end item
\item{ 
\index{setMap(TiledMap)}
{\bf  setMap}\\
\texttt{public void\ {\bf  setMap}(\texttt{com.badlogic.gdx.maps.tiled.TiledMap} {\bf  map})
\label{com.retroMachines.ui.screens.game.GameScreen.setMap(com.badlogic.gdx.maps.tiled.TiledMap)}}%end signature
}%end item
\end{itemize}
}
\subsection{Members inherited from class AbstractScreen }{
\texttt{com.retroMachines.ui.screens.AbstractScreen} {\small 
\refdefined{com.retroMachines.ui.screens.AbstractScreen}}
{\small 

\vskip -2em
\begin{itemize}
\item{\vskip -1.5ex 
\texttt{public void {\bf  dispose}()
}%end signature
}%end item
\item{\vskip -1.5ex 
\texttt{public void {\bf  hide}()
}%end signature
}%end item
\item{\vskip -1.5ex 
\texttt{public void {\bf  pause}()
}%end signature
}%end item
\item{\vskip -1.5ex 
\texttt{public void {\bf  render}(\texttt{float} {\bf  delta})
}%end signature
}%end item
\item{\vskip -1.5ex 
\texttt{public void {\bf  resize}(\texttt{int} {\bf  width},
\texttt{int} {\bf  height})
}%end signature
}%end item
\item{\vskip -1.5ex 
\texttt{public void {\bf  resume}()
}%end signature
}%end item
\item{\vskip -1.5ex 
\texttt{public void {\bf  show}()
}%end signature
}%end item
\end{itemize}
}
}
}
\chapter{Package com.retroMachines.util}{
\label{com.retroMachines.util}\hskip -.05in
\hbox to \hsize{\textit{ Package Contents\hfil Page}}
\vskip .13in
\hbox{{\bf  Classes}}
\entityintro{Constants}{com.retroMachines.util.Constants}{}
\vskip .1in
\vskip .1in
\section{\label{com.retroMachines.util.Constants}\index{Constants}Class Constants}{
\vskip .1in 
\subsection{Declaration}{
\small public class Constants
\\ {\bf  extends} java.lang.Object
\refdefined{java.lang.Object}}
\subsection{Constructor summary}{
\begin{verse}
{\bf Constants()} \\
\end{verse}
}
\subsection{Constructors}{
\vskip -2em
\begin{itemize}
\item{ 
\index{Constants()}
{\bf  Constants}\\
\texttt{public\ {\bf  Constants}()
\label{com.retroMachines.util.Constants()}}%end signature
}%end item
\end{itemize}
}
}
}
\chapter{Package com.retroMachines.game.controllers}{
\label{com.retroMachines.game.controllers}\hskip -.05in
\hbox to \hsize{\textit{ Package Contents\hfil Page}}
\vskip .13in
\hbox{{\bf  Classes}}
\entityintro{GameController}{com.retroMachines.game.controllers.GameController}{GameController This class represents the controller for the actual game.}
\entityintro{ProfileController}{com.retroMachines.game.controllers.ProfileController}{}
\entityintro{SettingController}{com.retroMachines.game.controllers.SettingController}{SettingsController}
\entityintro{StatisticController}{com.retroMachines.game.controllers.StatisticController}{}
\vskip .1in
\vskip .1in
\section{\label{com.retroMachines.game.controllers.GameController}\index{GameController}Class GameController}{
\vskip .1in 
GameController This class represents the controller for the actual game. It sets up levels and also disposes them afterwards. It saves progress to the persistent storage.\vskip .1in 
\subsection{Declaration}{
\small public class GameController
\\ {\bf  extends} java.lang.Object
\refdefined{java.lang.Object}}
\subsection{Constructor summary}{
\begin{verse}
{\bf GameController(RetroMachines)} \\
\end{verse}
}
\subsection{Method summary}{
\begin{verse}
{\bf getRetroMan()} returns the RetroMan instance\\
{\bf jumpRetroMan()} delegates a jump call to the retroMan\\
{\bf levelFinished()} this method will be called once a level has been complete including the evaluation.\\
{\bf startLevel(int)} sets initializes a given level and fires it up\\
\end{verse}
}
\subsection{Constructors}{
\vskip -2em
\begin{itemize}
\item{ 
\index{GameController(RetroMachines)}
{\bf  GameController}\\
\texttt{public\ {\bf  GameController}(\texttt{com.retroMachines.RetroMachines} {\bf  game})
\label{com.retroMachines.game.controllers.GameController(com.retroMachines.RetroMachines)}}%end signature
}%end item
\end{itemize}
}
\subsection{Methods}{
\vskip -2em
\begin{itemize}
\item{ 
\index{getRetroMan()}
{\bf  getRetroMan}\\
\texttt{public com.retroMachines.game.gameelements.RetroMan\ {\bf  getRetroMan}()
\label{com.retroMachines.game.controllers.GameController.getRetroMan()}}%end signature
\begin{itemize}
\item{
{\bf  Description}

returns the RetroMan instance
}
\item{{\bf  Returns} -- 
 
}%end item
\end{itemize}
}%end item
\item{ 
\index{jumpRetroMan()}
{\bf  jumpRetroMan}\\
\texttt{public void\ {\bf  jumpRetroMan}()
\label{com.retroMachines.game.controllers.GameController.jumpRetroMan()}}%end signature
\begin{itemize}
\item{
{\bf  Description}

delegates a jump call to the retroMan
}
\end{itemize}
}%end item
\item{ 
\index{levelFinished()}
{\bf  levelFinished}\\
\texttt{public void\ {\bf  levelFinished}()
\label{com.retroMachines.game.controllers.GameController.levelFinished()}}%end signature
\begin{itemize}
\item{
{\bf  Description}

this method will be called once a level has been complete including the evaluation. Afterwards the LevelMenuScreen will be shown to the user
}
\end{itemize}
}%end item
\item{ 
\index{startLevel(int)}
{\bf  startLevel}\\
\texttt{public void\ {\bf  startLevel}(\texttt{int} {\bf  levelId})
\label{com.retroMachines.game.controllers.GameController.startLevel(int)}}%end signature
\begin{itemize}
\item{
{\bf  Description}

sets initializes a given level and fires it up
}
\item{
{\bf  Parameters}
  \begin{itemize}
   \item{
\texttt{levelId} -- the level to be started}
  \end{itemize}
}%end item
\end{itemize}
}%end item
\end{itemize}
}
}
\section{\label{com.retroMachines.game.controllers.ProfileController}\index{ProfileController}Class ProfileController}{
\vskip .1in 
\subsection{Declaration}{
\small public class ProfileController
\\ {\bf  extends} java.lang.Object
\refdefined{java.lang.Object}}
\subsection{Field summary}{
\begin{verse}
{\bf MAX\_PROFILE\_NUMBER} the amount of profiles allowed in the game\\
\end{verse}
}
\subsection{Constructor summary}{
\begin{verse}
{\bf ProfileController(RetroMachines)} creates a new instance of the profile controller\\
\end{verse}
}
\subsection{Method summary}{
\begin{verse}
{\bf changeActiveProfile(String)} changes to the current profile to another profile\\
{\bf deleteCurrentProfile()} removes the currently active profile\\
{\bf getProfile()} \\
{\bf getProfileName()} Get the name of the currently active user\\
{\bf getProfileNames()} \\
{\bf isValidUsername(String)} checks if a given username is valid, meaning it is not occupied by another profile already\\
\end{verse}
}
\subsection{Fields}{
\begin{itemize}
\item{
\index{MAX\_PROFILE\_NUMBER}
\label{com.retroMachines.game.controllers.ProfileController.MAX_PROFILE_NUMBER}public static final int {\bf  MAX\_PROFILE\_NUMBER}\begin{itemize}
\item{\vskip -.9ex 
the amount of profiles allowed in the game}
\end{itemize}
}
\end{itemize}
}
\subsection{Constructors}{
\vskip -2em
\begin{itemize}
\item{ 
\index{ProfileController(RetroMachines)}
{\bf  ProfileController}\\
\texttt{public\ {\bf  ProfileController}(\texttt{com.retroMachines.RetroMachines} {\bf  game})
\label{com.retroMachines.game.controllers.ProfileController(com.retroMachines.RetroMachines)}}%end signature
\begin{itemize}
\item{
{\bf  Description}

creates a new instance of the profile controller
}
\item{
{\bf  Parameters}
  \begin{itemize}
   \item{
\texttt{game} -- the game for calls towards the game}
  \end{itemize}
}%end item
\end{itemize}
}%end item
\end{itemize}
}
\subsection{Methods}{
\vskip -2em
\begin{itemize}
\item{ 
\index{changeActiveProfile(String)}
{\bf  changeActiveProfile}\\
\texttt{public void\ {\bf  changeActiveProfile}(\texttt{java.lang.String} {\bf  profileName})
\label{com.retroMachines.game.controllers.ProfileController.changeActiveProfile(java.lang.String)}}%end signature
\begin{itemize}
\item{
{\bf  Description}

changes to the current profile to another profile
}
\item{
{\bf  Parameters}
  \begin{itemize}
   \item{
\texttt{profileName} -- the name of the profile}
  \end{itemize}
}%end item
\end{itemize}
}%end item
\item{ 
\index{deleteCurrentProfile()}
{\bf  deleteCurrentProfile}\\
\texttt{public void\ {\bf  deleteCurrentProfile}()
\label{com.retroMachines.game.controllers.ProfileController.deleteCurrentProfile()}}%end signature
\begin{itemize}
\item{
{\bf  Description}

removes the currently active profile
}
\end{itemize}
}%end item
\item{ 
\index{getProfile()}
{\bf  getProfile}\\
\texttt{public com.retroMachines.data.models.Profile\ {\bf  getProfile}()
\label{com.retroMachines.game.controllers.ProfileController.getProfile()}}%end signature
\begin{itemize}
\item{{\bf  Returns} -- 
the profile 
}%end item
\end{itemize}
}%end item
\item{ 
\index{getProfileName()}
{\bf  getProfileName}\\
\texttt{public java.lang.String\ {\bf  getProfileName}()
\label{com.retroMachines.game.controllers.ProfileController.getProfileName()}}%end signature
\begin{itemize}
\item{
{\bf  Description}

Get the name of the currently active user
}
\item{{\bf  Returns} -- 
The name of the currently active user; Empty String if no user is active. 
}%end item
\end{itemize}
}%end item
\item{ 
\index{getProfileNames()}
{\bf  getProfileNames}\\
\texttt{public java.lang.String\lbrack \rbrack \ {\bf  getProfileNames}()
\label{com.retroMachines.game.controllers.ProfileController.getProfileNames()}}%end signature
}%end item
\item{ 
\index{isValidUsername(String)}
{\bf  isValidUsername}\\
\texttt{public boolean\ {\bf  isValidUsername}(\texttt{java.lang.String} {\bf  username})
\label{com.retroMachines.game.controllers.ProfileController.isValidUsername(java.lang.String)}}%end signature
\begin{itemize}
\item{
{\bf  Description}

checks if a given username is valid, meaning it is not occupied by another profile already
}
\end{itemize}
}%end item
\end{itemize}
}
}
\section{\label{com.retroMachines.game.controllers.SettingController}\index{SettingController}Class SettingController}{
\vskip .1in 
SettingsController\vskip .1in 
\subsection{Declaration}{
\small public class SettingController
\\ {\bf  extends} java.lang.Object
\refdefined{java.lang.Object}}
\subsection{Constructor summary}{
\begin{verse}
{\bf SettingController(RetroMachines)} \\
\end{verse}
}
\subsection{Constructors}{
\vskip -2em
\begin{itemize}
\item{ 
\index{SettingController(RetroMachines)}
{\bf  SettingController}\\
\texttt{public\ {\bf  SettingController}(\texttt{com.retroMachines.RetroMachines} {\bf  game})
\label{com.retroMachines.game.controllers.SettingController(com.retroMachines.RetroMachines)}}%end signature
}%end item
\end{itemize}
}
}
\section{\label{com.retroMachines.game.controllers.StatisticController}\index{StatisticController}Class StatisticController}{
\vskip .1in 
\subsection{Declaration}{
\small public class StatisticController
\\ {\bf  extends} java.lang.Object
\refdefined{java.lang.Object}}
\subsection{Constructor summary}{
\begin{verse}
{\bf StatisticController(RetroMachines)} \\
\end{verse}
}
\subsection{Constructors}{
\vskip -2em
\begin{itemize}
\item{ 
\index{StatisticController(RetroMachines)}
{\bf  StatisticController}\\
\texttt{public\ {\bf  StatisticController}(\texttt{com.retroMachines.RetroMachines} {\bf  game})
\label{com.retroMachines.game.controllers.StatisticController(com.retroMachines.RetroMachines)}}%end signature
}%end item
\end{itemize}
}
}
}
\chapter{Package com.retroMachines.game.gameelements}{
\label{com.retroMachines.game.gameelements}\hskip -.05in
\hbox to \hsize{\textit{ Package Contents\hfil Page}}
\vskip .13in
\hbox{{\bf  Classes}}
\entityintro{GameElement}{com.retroMachines.game.gameelements.GameElement}{}
\entityintro{LightElement}{com.retroMachines.game.gameelements.LightElement}{}
\entityintro{MachineElement}{com.retroMachines.game.gameelements.MachineElement}{}
\entityintro{MetalElement}{com.retroMachines.game.gameelements.MetalElement}{}
\entityintro{RetroMan}{com.retroMachines.game.gameelements.RetroMan}{}
\vskip .1in
\vskip .1in
\section{\label{com.retroMachines.game.gameelements.GameElement}\index{GameElement}Class GameElement}{
\vskip .1in 
\subsection{Declaration}{
\small public abstract class GameElement
\\ {\bf  extends} java.lang.Object
\refdefined{java.lang.Object}}
\subsection{All known subclasses}{MachineElement\small{\refdefined{com.retroMachines.game.gameelements.MachineElement}}, MetalElement\small{\refdefined{com.retroMachines.game.gameelements.MetalElement}}, LightElement\small{\refdefined{com.retroMachines.game.gameelements.LightElement}}}
\subsection{Constructor summary}{
\begin{verse}
{\bf GameElement()} \\
\end{verse}
}
\subsection{Method summary}{
\begin{verse}
{\bf render(float)} \\
\end{verse}
}
\subsection{Constructors}{
\vskip -2em
\begin{itemize}
\item{ 
\index{GameElement()}
{\bf  GameElement}\\
\texttt{public\ {\bf  GameElement}()
\label{com.retroMachines.game.gameelements.GameElement()}}%end signature
}%end item
\end{itemize}
}
\subsection{Methods}{
\vskip -2em
\begin{itemize}
\item{ 
\index{render(float)}
{\bf  render}\\
\texttt{public abstract void\ {\bf  render}(\texttt{float} {\bf  deltaTime})
\label{com.retroMachines.game.gameelements.GameElement.render(float)}}%end signature
}%end item
\end{itemize}
}
}
\section{\label{com.retroMachines.game.gameelements.LightElement}\index{LightElement}Class LightElement}{
\vskip .1in 
\subsection{Declaration}{
\small public class LightElement
\\ {\bf  extends} com.retroMachines.game.gameelements.GameElement
\refdefined{com.retroMachines.game.gameelements.GameElement}}
\subsection{Constructor summary}{
\begin{verse}
{\bf LightElement()} \\
\end{verse}
}
\subsection{Method summary}{
\begin{verse}
{\bf render(float)} \\
\end{verse}
}
\subsection{Constructors}{
\vskip -2em
\begin{itemize}
\item{ 
\index{LightElement()}
{\bf  LightElement}\\
\texttt{public\ {\bf  LightElement}()
\label{com.retroMachines.game.gameelements.LightElement()}}%end signature
}%end item
\end{itemize}
}
\subsection{Methods}{
\vskip -2em
\begin{itemize}
\item{ 
\index{render(float)}
{\bf  render}\\
\texttt{public abstract void\ {\bf  render}(\texttt{float} {\bf  deltaTime})
\label{com.retroMachines.game.gameelements.LightElement.render(float)}}%end signature
}%end item
\end{itemize}
}
\subsection{Members inherited from class GameElement }{
\texttt{com.retroMachines.game.gameelements.GameElement} {\small 
\refdefined{com.retroMachines.game.gameelements.GameElement}}
{\small 

\vskip -2em
\begin{itemize}
\item{\vskip -1.5ex 
\texttt{public abstract void {\bf  render}(\texttt{float} {\bf  deltaTime})
}%end signature
}%end item
\end{itemize}
}
}
\section{\label{com.retroMachines.game.gameelements.MachineElement}\index{MachineElement}Class MachineElement}{
\vskip .1in 
\subsection{Declaration}{
\small public class MachineElement
\\ {\bf  extends} com.retroMachines.game.gameelements.GameElement
\refdefined{com.retroMachines.game.gameelements.GameElement}}
\subsection{Constructor summary}{
\begin{verse}
{\bf MachineElement()} \\
\end{verse}
}
\subsection{Method summary}{
\begin{verse}
{\bf render(float)} \\
\end{verse}
}
\subsection{Constructors}{
\vskip -2em
\begin{itemize}
\item{ 
\index{MachineElement()}
{\bf  MachineElement}\\
\texttt{public\ {\bf  MachineElement}()
\label{com.retroMachines.game.gameelements.MachineElement()}}%end signature
}%end item
\end{itemize}
}
\subsection{Methods}{
\vskip -2em
\begin{itemize}
\item{ 
\index{render(float)}
{\bf  render}\\
\texttt{public abstract void\ {\bf  render}(\texttt{float} {\bf  deltaTime})
\label{com.retroMachines.game.gameelements.MachineElement.render(float)}}%end signature
}%end item
\end{itemize}
}
\subsection{Members inherited from class GameElement }{
\texttt{com.retroMachines.game.gameelements.GameElement} {\small 
\refdefined{com.retroMachines.game.gameelements.GameElement}}
{\small 

\vskip -2em
\begin{itemize}
\item{\vskip -1.5ex 
\texttt{public abstract void {\bf  render}(\texttt{float} {\bf  deltaTime})
}%end signature
}%end item
\end{itemize}
}
}
\section{\label{com.retroMachines.game.gameelements.MetalElement}\index{MetalElement}Class MetalElement}{
\vskip .1in 
\subsection{Declaration}{
\small public class MetalElement
\\ {\bf  extends} com.retroMachines.game.gameelements.GameElement
\refdefined{com.retroMachines.game.gameelements.GameElement}}
\subsection{Constructor summary}{
\begin{verse}
{\bf MetalElement()} \\
\end{verse}
}
\subsection{Method summary}{
\begin{verse}
{\bf render(float)} \\
\end{verse}
}
\subsection{Constructors}{
\vskip -2em
\begin{itemize}
\item{ 
\index{MetalElement()}
{\bf  MetalElement}\\
\texttt{public\ {\bf  MetalElement}()
\label{com.retroMachines.game.gameelements.MetalElement()}}%end signature
}%end item
\end{itemize}
}
\subsection{Methods}{
\vskip -2em
\begin{itemize}
\item{ 
\index{render(float)}
{\bf  render}\\
\texttt{public abstract void\ {\bf  render}(\texttt{float} {\bf  deltaTime})
\label{com.retroMachines.game.gameelements.MetalElement.render(float)}}%end signature
}%end item
\end{itemize}
}
\subsection{Members inherited from class GameElement }{
\texttt{com.retroMachines.game.gameelements.GameElement} {\small 
\refdefined{com.retroMachines.game.gameelements.GameElement}}
{\small 

\vskip -2em
\begin{itemize}
\item{\vskip -1.5ex 
\texttt{public abstract void {\bf  render}(\texttt{float} {\bf  deltaTime})
}%end signature
}%end item
\end{itemize}
}
}
\section{\label{com.retroMachines.game.gameelements.RetroMan}\index{RetroMan}Class RetroMan}{
\vskip .1in 
\subsection{Declaration}{
\small public class RetroMan
\\ {\bf  extends} java.lang.Object
\refdefined{java.lang.Object}}
\subsection{Field summary}{
\begin{verse}
{\bf HEIGHT} \\
{\bf WIDTH} \\
\end{verse}
}
\subsection{Constructor summary}{
\begin{verse}
{\bf RetroMan()} \\
\end{verse}
}
\subsection{Method summary}{
\begin{verse}
{\bf canJump()} \\
{\bf getPos()} \\
{\bf hasPickedUpElement()} \\
{\bf jump()} \\
{\bf landed()} Call this method when the character is supposed to jump\\
{\bf pickupElement(GameElement)} \\
{\bf render(float)} \\
\end{verse}
}
\subsection{Fields}{
\begin{itemize}
\item{
\index{WIDTH}
\label{com.retroMachines.game.gameelements.RetroMan.WIDTH}public static final float {\bf  WIDTH}}
\item{
\index{HEIGHT}
\label{com.retroMachines.game.gameelements.RetroMan.HEIGHT}public static final float {\bf  HEIGHT}}
\end{itemize}
}
\subsection{Constructors}{
\vskip -2em
\begin{itemize}
\item{ 
\index{RetroMan()}
{\bf  RetroMan}\\
\texttt{public\ {\bf  RetroMan}()
\label{com.retroMachines.game.gameelements.RetroMan()}}%end signature
}%end item
\end{itemize}
}
\subsection{Methods}{
\vskip -2em
\begin{itemize}
\item{ 
\index{canJump()}
{\bf  canJump}\\
\texttt{public boolean\ {\bf  canJump}()
\label{com.retroMachines.game.gameelements.RetroMan.canJump()}}%end signature
}%end item
\item{ 
\index{getPos()}
{\bf  getPos}\\
\texttt{public com.badlogic.gdx.math.Vector2\ {\bf  getPos}()
\label{com.retroMachines.game.gameelements.RetroMan.getPos()}}%end signature
}%end item
\item{ 
\index{hasPickedUpElement()}
{\bf  hasPickedUpElement}\\
\texttt{public boolean\ {\bf  hasPickedUpElement}()
\label{com.retroMachines.game.gameelements.RetroMan.hasPickedUpElement()}}%end signature
\begin{itemize}
\item{{\bf  Returns} -- 
 
}%end item
\end{itemize}
}%end item
\item{ 
\index{jump()}
{\bf  jump}\\
\texttt{public void\ {\bf  jump}()
\label{com.retroMachines.game.gameelements.RetroMan.jump()}}%end signature
}%end item
\item{ 
\index{landed()}
{\bf  landed}\\
\texttt{public void\ {\bf  landed}()
\label{com.retroMachines.game.gameelements.RetroMan.landed()}}%end signature
\begin{itemize}
\item{
{\bf  Description}

Call this method when the character is supposed to jump
}
\end{itemize}
}%end item
\item{ 
\index{pickupElement(GameElement)}
{\bf  pickupElement}\\
\texttt{public void\ {\bf  pickupElement}(\texttt{GameElement} {\bf  element})
\label{com.retroMachines.game.gameelements.RetroMan.pickupElement(com.retroMachines.game.gameelements.GameElement)}}%end signature
\begin{itemize}
\item{
{\bf  Parameters}
  \begin{itemize}
   \item{
\texttt{element} -- }
  \end{itemize}
}%end item
\end{itemize}
}%end item
\item{ 
\index{render(float)}
{\bf  render}\\
\texttt{public com.badlogic.gdx.graphics.g2d.TextureRegion\ {\bf  render}(\texttt{float} {\bf  deltaTime})
\label{com.retroMachines.game.gameelements.RetroMan.render(float)}}%end signature
\begin{itemize}
\item{
{\bf  Parameters}
  \begin{itemize}
   \item{
\texttt{deltaTime} -- }
  \end{itemize}
}%end item
\end{itemize}
}%end item
\end{itemize}
}
}
}
\chapter{Package com.retroMachines.util.lambda}{
\label{com.retroMachines.util.lambda}\hskip -.05in
\hbox to \hsize{\textit{ Package Contents\hfil Page}}
\vskip .13in
\hbox{{\bf  Classes}}
\entityintro{Abstraction}{com.retroMachines.util.lambda.Abstraction}{}
\entityintro{Application}{com.retroMachines.util.lambda.Application}{}
\entityintro{Tree}{com.retroMachines.util.lambda.Tree}{}
\entityintro{Variable}{com.retroMachines.util.lambda.Variable}{}
\entityintro{Vertex}{com.retroMachines.util.lambda.Vertex}{}
\vskip .1in
\vskip .1in
\section{\label{com.retroMachines.util.lambda.Abstraction}\index{Abstraction}Class Abstraction}{
\vskip .1in 
\subsection{Declaration}{
\small public class Abstraction
\\ {\bf  extends} com.retroMachines.util.lambda.Vertex
\refdefined{com.retroMachines.util.lambda.Vertex}}
\subsection{Constructor summary}{
\begin{verse}
{\bf Abstraction(char)} constructor\\
\end{verse}
}
\subsection{Method summary}{
\begin{verse}
{\bf alphaConversion()} fulfills alpha conversion.\\
{\bf betaReduction()} fulfills one step of beta-reduction.\\
{\bf getID()} getter for id\\
{\bf setID(char)} setter for id.\\
\end{verse}
}
\subsection{Constructors}{
\vskip -2em
\begin{itemize}
\item{ 
\index{Abstraction(char)}
{\bf  Abstraction}\\
\texttt{public\ {\bf  Abstraction}(\texttt{char} {\bf  id})
\label{com.retroMachines.util.lambda.Abstraction(char)}}%end signature
\begin{itemize}
\item{
{\bf  Description}

constructor
}
\item{
{\bf  Parameters}
  \begin{itemize}
   \item{
\texttt{id} -- id to set}
  \end{itemize}
}%end item
\end{itemize}
}%end item
\end{itemize}
}
\subsection{Methods}{
\vskip -2em
\begin{itemize}
\item{ 
\index{alphaConversion()}
{\bf  alphaConversion}\\
\texttt{public boolean\ {\bf  alphaConversion}()
\label{com.retroMachines.util.lambda.Abstraction.alphaConversion()}}%end signature
\begin{itemize}
\item{
{\bf  Description}

fulfills alpha conversion. Makes sure that all vertices have unique id's.
}
\item{{\bf  Returns} -- 
true if at least one id changed, false if no id changed. 
}%end item
\end{itemize}
}%end item
\item{ 
\index{betaReduction()}
{\bf  betaReduction}\\
\texttt{public boolean\ {\bf  betaReduction}()
\label{com.retroMachines.util.lambda.Abstraction.betaReduction()}}%end signature
\begin{itemize}
\item{
{\bf  Description}

fulfills one step of beta-reduction.
}
\item{{\bf  Returns} -- 
true if this abstraction has changed, false otherwise 
}%end item
\end{itemize}
}%end item
\item{ 
\index{getID()}
{\bf  getID}\\
\texttt{public char\ {\bf  getID}()
\label{com.retroMachines.util.lambda.Abstraction.getID()}}%end signature
\begin{itemize}
\item{
{\bf  Description}

getter for id
}
\item{{\bf  Returns} -- 
id 
}%end item
\end{itemize}
}%end item
\item{ 
\index{setID(char)}
{\bf  setID}\\
\texttt{public void\ {\bf  setID}(\texttt{char} {\bf  id})
\label{com.retroMachines.util.lambda.Abstraction.setID(char)}}%end signature
\begin{itemize}
\item{
{\bf  Description}

setter for id.
}
\item{
{\bf  Parameters}
  \begin{itemize}
   \item{
\texttt{id} -- to set.}
  \end{itemize}
}%end item
\end{itemize}
}%end item
\end{itemize}
}
}
\section{\label{com.retroMachines.util.lambda.Application}\index{Application}Class Application}{
\vskip .1in 
\subsection{Declaration}{
\small public class Application
\\ {\bf  extends} com.retroMachines.util.lambda.Vertex
\refdefined{com.retroMachines.util.lambda.Vertex}}
\subsection{Constructor summary}{
\begin{verse}
{\bf Application()} constructor sets empty idList, isChecked to false\\
\end{verse}
}
\subsection{Method summary}{
\begin{verse}
{\bf setChecked()} sets isChecked to true\\
\end{verse}
}
\subsection{Constructors}{
\vskip -2em
\begin{itemize}
\item{ 
\index{Application()}
{\bf  Application}\\
\texttt{public\ {\bf  Application}()
\label{com.retroMachines.util.lambda.Application()}}%end signature
\begin{itemize}
\item{
{\bf  Description}

constructor sets empty idList, isChecked to false
}
\end{itemize}
}%end item
\end{itemize}
}
\subsection{Methods}{
\vskip -2em
\begin{itemize}
\item{ 
\index{setChecked()}
{\bf  setChecked}\\
\texttt{public void\ {\bf  setChecked}()
\label{com.retroMachines.util.lambda.Application.setChecked()}}%end signature
\begin{itemize}
\item{
{\bf  Description}

sets isChecked to true
}
\end{itemize}
}%end item
\end{itemize}
}
}
\section{\label{com.retroMachines.util.lambda.Tree}\index{Tree}Class Tree}{
\vskip .1in 
\subsection{Declaration}{
\small public class Tree
\\ {\bf  extends} java.lang.Object
\refdefined{java.lang.Object}}
\subsection{Constructor summary}{
\begin{verse}
{\bf Tree(String)} creates Tree representation of given lambda-term in string representation\\
{\bf Tree(Vertex)} creates Tree object of given vertex.\\
\end{verse}
}
\subsection{Method summary}{
\begin{verse}
{\bf getStart()} getter for start\\
\end{verse}
}
\subsection{Constructors}{
\vskip -2em
\begin{itemize}
\item{ 
\index{Tree(String)}
{\bf  Tree}\\
\texttt{public\ {\bf  Tree}(\texttt{java.lang.String} {\bf  term})
\label{com.retroMachines.util.lambda.Tree(java.lang.String)}}%end signature
\begin{itemize}
\item{
{\bf  Description}

creates Tree representation of given lambda-term in string representation
}
\item{
{\bf  Parameters}
  \begin{itemize}
   \item{
\texttt{term} -- string representation of labmda-term}
  \end{itemize}
}%end item
\end{itemize}
}%end item
\item{ 
\index{Tree(Vertex)}
{\bf  Tree}\\
\texttt{public\ {\bf  Tree}(\texttt{Vertex} {\bf  start})
\label{com.retroMachines.util.lambda.Tree(com.retroMachines.util.lambda.Vertex)}}%end signature
\begin{itemize}
\item{
{\bf  Description}

creates Tree object of given vertex.
}
\item{
{\bf  Parameters}
  \begin{itemize}
   \item{
\texttt{start} -- root of tree to create}
  \end{itemize}
}%end item
\end{itemize}
}%end item
\end{itemize}
}
\subsection{Methods}{
\vskip -2em
\begin{itemize}
\item{ 
\index{getStart()}
{\bf  getStart}\\
\texttt{public Vertex\ {\bf  getStart}()
\label{com.retroMachines.util.lambda.Tree.getStart()}}%end signature
\begin{itemize}
\item{
{\bf  Description}

getter for start
}
\item{{\bf  Returns} -- 
start 
}%end item
\end{itemize}
}%end item
\end{itemize}
}
}
\section{\label{com.retroMachines.util.lambda.Variable}\index{Variable}Class Variable}{
\vskip .1in 
\subsection{Declaration}{
\small public class Variable
\\ {\bf  extends} com.retroMachines.util.lambda.Vertex
\refdefined{com.retroMachines.util.lambda.Vertex}}
\subsection{Constructor summary}{
\begin{verse}
{\bf Variable(char)} constructor\\
\end{verse}
}
\subsection{Method summary}{
\begin{verse}
{\bf getId()} getter for id.\\
{\bf setId(char)} setter for id\\
\end{verse}
}
\subsection{Constructors}{
\vskip -2em
\begin{itemize}
\item{ 
\index{Variable(char)}
{\bf  Variable}\\
\texttt{public\ {\bf  Variable}(\texttt{char} {\bf  id})
\label{com.retroMachines.util.lambda.Variable(char)}}%end signature
\begin{itemize}
\item{
{\bf  Description}

constructor
}
\item{
{\bf  Parameters}
  \begin{itemize}
   \item{
\texttt{id} -- id to set}
  \end{itemize}
}%end item
\end{itemize}
}%end item
\end{itemize}
}
\subsection{Methods}{
\vskip -2em
\begin{itemize}
\item{ 
\index{getId()}
{\bf  getId}\\
\texttt{public int\ {\bf  getId}()
\label{com.retroMachines.util.lambda.Variable.getId()}}%end signature
\begin{itemize}
\item{
{\bf  Description}

getter for id.
}
\item{{\bf  Returns} -- 
 
}%end item
\end{itemize}
}%end item
\item{ 
\index{setId(char)}
{\bf  setId}\\
\texttt{public void\ {\bf  setId}(\texttt{char} {\bf  id})
\label{com.retroMachines.util.lambda.Variable.setId(char)}}%end signature
\begin{itemize}
\item{
{\bf  Description}

setter for id
}
\item{
{\bf  Parameters}
  \begin{itemize}
   \item{
\texttt{id} -- id to set}
  \end{itemize}
}%end item
\end{itemize}
}%end item
\end{itemize}
}
}
\section{\label{com.retroMachines.util.lambda.Vertex}\index{Vertex}Class Vertex}{
\vskip .1in 
\subsection{Declaration}{
\small public abstract class Vertex
\\ {\bf  extends} java.lang.Object
\refdefined{java.lang.Object}}
\subsection{All known subclasses}{Application\small{\refdefined{com.retroMachines.util.lambda.Application}}, Abstraction\small{\refdefined{com.retroMachines.util.lambda.Abstraction}}, Variable\small{\refdefined{com.retroMachines.util.lambda.Variable}}}
\subsection{Constructor summary}{
\begin{verse}
{\bf Vertex()} \\
\end{verse}
}
\subsection{Constructors}{
\vskip -2em
\begin{itemize}
\item{ 
\index{Vertex()}
{\bf  Vertex}\\
\texttt{public\ {\bf  Vertex}()
\label{com.retroMachines.util.lambda.Vertex()}}%end signature
}%end item
\end{itemize}
}
}
}
\printindex
\end{document}
