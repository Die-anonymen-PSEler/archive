\documentclass[parskip=full]{scrreprt}

\usepackage{helvet}
\usepackage[T1]{fontenc}
\usepackage[ngerman]{babel}
\usepackage[utf8]{inputenc}
\inputencoding{utf8}
\usepackage{hyperref}
\usepackage{mathtools}
\usepackage[xindy]{imakeidx}
\makeindex
\usepackage{enumitem}
\usepackage{graphicx}
\usepackage{cleveref}
\usepackage[autostyle=true,german=quotes]{csquotes}
\renewcommand{\familydefault}{\sfdefault}
\begin{document}


\title{Implementierungsbericht: RetroMachines, \\ RetroFactory}
\author{Luca Becker, Henrike Hardt,\\Larissa Schmid, Adrian Schulte,\\Maik Wiesner}
\date{1. Dezember 2014}
\maketitle
\thispagestyle{empty}

\clearpage

\thispagestyle{empty}
\tableofcontents
\thispagestyle{empty}

\clearpage
\setcounter{page}{1}

\chapter{Einleitung}

\enquote{RetroMachines}

\chapter{Verbesserungen}

Das Spiel ist genauso super als wie vorher. Haben kleine ostdeutsche Kinder testen lassen. Waren überrascht von der Technik. Unfähig zu spielen. Schieben das auf Kinder.

\chapter{Codequalität}

Durch den Einsatz verschiedener Tools haben wir die Codequalität unseres Codes weiter verbessern können.

\section{CodePro AnalytiX}



\section{FindBugs}



\chapter{Unittests}

Wir verwenden jUnit-Tests um unsere Klassen einer Qualitätsprüfung zu unterziehen. Zum Ende der Qualitätssicherungsphase sind über 300 jUnit-Tests.

\chapter{Integrationstest}

\section{Controller}

\section{Spiel spielen}

\chapter{Systemtests}

\chapter{Überdeckung}

\begin{tabular} { | l | c | c | c | c | c | c | }
	\hline
	\textbf{Package} & \textbf{IC} & \textbf{BC} & \textbf{CI} & \textbf{MI} & \textbf{CB} & \textbf{MB} \\
	\hline
	Insgesamt & 81,3\% & 60\% & 11047 & 2547 & 420 & 270 \\
	\hline
	com.retroMachines \\
	\hline
	com.retroMachines.data & 100\% & 100\% & 205 & 0 & 6 & 0  \\
	\hline
	com.retroMachines.data.models & 100\% & 100\% & 603 & 0 & 16 & 0 \\
	\hline
	com.retroMachines.game \\
	\hline
	com.retroMachines.game.controllers \\
	\hline
	com.retroMachines.game.gameelements \\
	\hline
	com.retroMachines.ui & 100\% & 100\% & 323 & 0 & 8 & 0 \\
	\hline
	com.retroMachines.ui.screens & 100\% & 100\% & 51 & 0 & 0 & 0 \\
	\hline
	com.retroMachines.ui.screens.game \\
	\hline
	com.retroMachines.ui.screens.menus & 99\% & 85\% & 3965 & 50 & 41 & 7 \\
	\hline
	com.retroMachines.util \\
	\hline
	com.retroMachines.util.lambda \\
	\hline
\end{tabular}

Legende: \\
	IC = Instruction Coverage \\
	BC = Branch Coverage \\
	CI = Covered Instruction \\
	MI = Missed Instruction \\
	CB = Covered Branches \\
	MB = Missed Branches \\
	n/a: In diesem Fall macht eine Branchabdeckung keinen Sinn da keine Abzweigungen innerhalb des Packages existieren. \\

In dem mit dem Dokument mitgeschickten EclEmma Coverage Report sind diese Daten und mehr in einem interaktiven Format zusammengefasst. 
Er erlaubt die eingehende Betrachtung der Überdeckung und ermöglicht es die Abdeckung in den einzelnen Paketen, Klassen und Methoden detailliert aufzulisten.

\chapter{Lasttests}

\section{Monkey Testing}

Wir haben unsere Gerät mehreren Monkey-Tests unterzogen. Dabei sind uns keinerlei Auffälligkeiten aufgefallen.

\chapter{Gefundene und behobene Bugs}

\begin{description}
	\item[Löschen des aktiven Profils:] Es wurde ein Fehler behoben der Auftrifft, falls der Benutzer das aktive Profil löscht.
	\item[Fehler beim Starten des Spiels: ] Es war möglich das Spiel zu nutzen, obwohl ein Tutorialhinweis angezeigt wurde.
\end{description}

\chapter{Testwerkzeuge}

\section{EclEmma}

EclEmma erlaubte es uns die Codecoverage unseres Projekt zu überprüfen. Darüber hinaus verfügt dieses Tool über die Möglichkeit eines HTML-Exports.

\section{jUnit}

jUnit erlaubt das Testen einzelner Bestandteile unseres Codes. Darüber hinaus waren wir durch jUnit in der Lage verschiedene Integrationstests auszuführen.

\end{document}