\documentclass[parskip=full]{scrreprt}

\usepackage{helvet}
\usepackage[T1]{fontenc}
\usepackage[ngerman]{babel}
\usepackage[utf8]{inputenc}
\inputencoding{utf8}
\usepackage{hyperref}
\usepackage{mathtools}
\usepackage[xindy]{imakeidx}
\makeindex
\usepackage{enumitem}
\usepackage{graphicx}
\usepackage{cleveref}
\usepackage[autostyle=true,german=quotes]{csquotes}
\renewcommand{\familydefault}{\sfdefault}
\begin{document}


\title{Implementierungsbericht: RetroMachines, \\ RetroFactory}
\author{Luca Becker, Henrike Hardt,\\Larissa Schmid, Adrian Schulte,\\Maik Wiesner}
\date{26. März 2015}
\maketitle
\thispagestyle{empty}

\clearpage

\thispagestyle{empty}
\tableofcontents
\thispagestyle{empty}

\clearpage
\setcounter{page}{1}

\chapter{Einleitung}

Nach der Implementierungsphase war bereits ein erster Prototyp von RetroMachines fertiggestellt, der zwar die Grundfunktionalität abdeckte, aber an einigen Stellen noch nicht ganz den Spezifizierungen im Pflichtenheft entsprach oder gar abstürzte bei Fehlverhalten.
Dem haben wir uns nun in der Phase der Qualitätssicherung angenommen, sodass wir die Applikation weiter verbessern sowie effizient testen konnten, um bis jetzt unentdeckte Fehler noch aufzuspüren und diesen bei anschließenden weiteren Änderungen vorzubeugen.
Wie wir dies umgesetzt haben, dokumentiert der folgende Testbericht. Er gibt zuerst einen kurzen Überblick über die Verbesserungen und vereinfachte Bedienabläufe im Vergleich zum Pflichtenheft, die wir erzielen konnten. Danach folgt ein Abschnitt über die Qualität unseres Code, bevor ein Überblick über die tatsächlich geschriebenen Tests, Testwerkzeuge und Fehler erfolgt.

\chapter{Verbesserungen}

Die Qualitätssicherungsphase hat sich größtenteils auf die Verbesserung des Codes, das Testen des Codes, sowie die Optimierung der Auswertung der Lambda-Ausdrücke konzentriert. Hier sowie an der Benutzeroberfläche des Spiels konnten signifikante Verbesserungen erzielt werden. 

\begin{description}
	\item[UI-Verbesserungen] Einige UI-Elemente haben eine Scroll-Funktion erhalten, was eine bessere Benutzung auf kleineren Displays erlaubt.
	\item[Erweiterung der Level] Es wurden weitere Level zum Spiel hinzugefügt, um dem Benutzer einen anhaltenden Spielspaß zu gewährleisten.
	\item[Profilsteuerung] Profilnamen, die sich nur durch Groß- und Kleinschreibung unterscheiden, sind nun nicht mehr möglich, um Verwirrung zu vermeiden.
	\item[Profilerstellung] Dem Benutzer werden nun Fehlermeldungen angezeigt, wenn die Erstellung eines Profils fehlschlägt beziehungsweise verwehrt wird.
	\item[Scrollen] Das Scrollen in X-Richtung auf einigen UI-Elementen wurde deaktiviert.
\end{description}

\chapter{Abweichungen vom Pflichtenheft}
Aufgrund der besseren Benutzbarkeit und Veranschaulichung des Lambda-Kalküls haben sich einige Änderungen im Ablauf zum Pflichtenheft ergeben.

\section{Globale Testfälle}
\begin{enumerate}
	\item Da die Applikation sehr schnell lädt, wurde der Ladebildschirm unnötig
	\item Die Einstellungen öffnen sich nicht, sobald ein Profil angelegt wurde. Es wird direkt auf den Hauptbildschirm weitergeleitet, von wo alle Möglichkeiten der Applikation direkt zur Verfügung stehen.
	\item Nach dem Laden wird automatisch das zuletzt aktive Profil wieder als aktiv gesetzt, es muss nicht mehr explizit bei jedem Spielstart ausgewählt werden, welches Profil man benutzen möchte.
	\item Das Tutorial wurde für einen vereinfachten Ablauf in die ersten Level des Spiels integriert und erscheint in Form von Dialogen. Danach kann direkt das ausgewählte Level gespielt werden, es wird nicht mehr auf den Hauptbildschirm zurückgeleitet. Beim erneuten Auswählen des Levels wird das Tutorial nicht mehr angezeigt, da es bereits absolviert wurde. Dies kann in den benutzerspezifischen Einstellungen jedoch zurückgesetzt werden, falls noch Unklarheiten bestehen.
	\item Legt man ein neues Profil an, so wird der gesamte Bildschirm zum Erstellen in einem statt zwei Bildschirmen angezeigt. Es muss außerdem nicht über Wischgesten an das Ende der Liste gescrollt werden, der "Benutzer anlegen"-Button befindet sich immer sichtbar auf dem unteren Teil der Profilübersicht.
	\item Um ein Benutzerprofil als aktiv auszuwählen, genügt es nicht, es nur durch Berühren auszuwählen. Es muss außerdem noch explizit der "Benutzer verwenden"-Button betätigt werden. Dies soll ungewollte Profilwechsel vermeiden.
	\item Möchte man ein Benutzerprofil bearbeiten, so muss dieses aktiv sein. Die benutzerspezifischen Einstellungen erreicht man nun nicht über den Benutzerverwaltungsbutton, sondern die Einstellungen und dort den weiteren "Benutzer bearbeiten"-Button.
	\item Im Pause-Dialog während des Levels werden nur die Möglichkeiten ins Level-Menü und zurück ins Spiel angegeben, nicht zusätzlich noch zu den Einstellungen und in das Hauptmenü.
\end{enumerate}

\section{Funktionale und nichtfunktionale Anforderungen}
\begin{enumerate}[resume]
	\item Da der Ladebildschirm nicht mehr vorhanden ist, werden auch während des Ladens keine Tipps zum Spiel angezeigt.
	\item Aufgrund der animierten Auswertung kann es in umfangreicheren Leveln zu Auswertungszeiten von über fünf Sekunden kommen.
\end{enumerate}

\chapter{Codequalität}

Durch den Einsatz verschiedener Tools haben wir die Qualität unseres Codes weiter verbessern können.

\section{CodePro AnalytiX}

CodePro AnalytiX bietet einen großen Umfang an Möglichkeiten zur statischen Codeanalyse. Wir konnten damit vor allem unseren Codestyle an Konventionen der Java-Programmierung durch Umbenennen einzelner Attribute und Methoden sowie Abändern einzelner Abfragen anpassen. Darüber hinaus haben wir toten Code entfernt sowie doppelte Abläufe im Programm in neue Methoden ausgelagert.
Außerdem wurde uns so komfortabel angezeigt, an welchen Stellen wir unsere Dokumentation durch JavaDoc noch verbessern mussten. Dies war zum Teil bei neu hinzugefügten Methoden der Fall, aber auch, wenn sich die Aufrufparameter geändert hatten.

\section{FindBugs}

Auch FindBugs dient zur statischen Codeanalyse und sucht nach häufigen Fehlermustern in Programmen. Hierbei konnten wir noch ein paar Verbesserungen am Codestyle vornehmen, wirkliche Fehler gab es jedoch nicht. 

\section{Strukturierung}
Zur Übersichtlichkeit des Codes haben wir uns dazu entschieden, den gesamten Code nach folgenden Regeln zu sortieren: \newline
Im Allgemeinen gilt die Reihenfolge "`abstract"', "`private"', "`protected"' und "`public"'. In Ausnahmefällen kann es zu Abweichungen kommen, zum Beispiel bei statischen Attributen oder Methoden:
\subsection{Attribute}
\begin{enumerate}
	\item[static final] (allgemeine Reihenfolge) 
	\item[static] (siehe Oben)
	\item[private]
	\item[protected]
	\item[pulic]
\end{enumerate}
\subsection{Methoden}
\begin{enumerate}
	\item[Konstruktoren] (in der oben gegebenen Reihenfolge)
	\item[abstract] (siehe Oben)
	\item[private] (static, normal)
	\item[protected] (static, overwritten, normal)
	\item[public] (static, overwritten, normal, getter and setter)
\end{enumerate}
\subsection{Subklassen}
Die Subklassen sind in der gleichen Reihenfolge aufgebaut wie normale Klassen.

\chapter{Testen}

\section{jUnit}

Das Programm wurde über 300 Unittests unterzogen. Hierbei kam jUnit zum Einsatz, welches besonders einfaches und effektives Testen erlaubt hat.

\subsection{Überdeckung}

\begin{tabular} { | l | c | c | c | c | c | c | }
	\hline
	\textbf{Package} & \textbf{IC} & \textbf{BC} & \textbf{CI} & \textbf{MI} & \textbf{CB} & \textbf{MB} \\
	\hline
	Insgesamt & 94,0\% & 78\% & 13241 & 849 & 574 & 162 \\
	\hline
	com.retroMachines & 100\% & 75\% & 106 & 0 & 3 & 1 \\
	\hline
	com.retroMachines.data & 100\% & 100\% & 205 & 0 & 6 & 0  \\
	\hline
	com.retroMachines.data.models & 100\% & 100\% & 603 & 0 & 16 & 0 \\
	\hline
	com.retroMachines.game & 93\% & 72\% & 786 & 59 & 59 & 23 \\
	\hline
	com.retroMachines.game.controllers & 98\% & 89\% & 1579 & 39 & 105 & 13 \\
	\hline
	com.retroMachines.game.gameelements & 96\% & 91\% & 639 & 28 & 43 & 4 \\
	\hline
	com.retroMachines.ui & 100\% & 100\% & 323 & 0 & 8 & 0 \\
	\hline
	com.retroMachines.ui.screens & 100\% & 100\% & 51 & 0 & 0 & 0 \\
	\hline
	com.retroMachines.ui.screens.game & 89\% & 56\% & 1956 & 247 & 29 & 23 \\
	\hline
	com.retroMachines.ui.screens.menus & 99\% & 85\% & 3965 & 50 & 41 & 7 \\
	\hline
	com.retroMachines.util & 100\% & 67\% & 94 & 0 & 4 & 2 \\
	\hline
	com.retroMachines.util.lambda & 87\% & 74\% & 2755 & 422 & 257 & 88\\
	\hline
\end{tabular}

\subsection*{Legende}
\begin{description}
	\item[IC] Instruction Coverage
	\item[BC] Branch Coverage
	\item[CI] Covered Instruction
	\item[MI] Missed Instruction
	\item[CB] Covered Branches
	\item[MB] Missed Branches
\end{description}	

In dem mit dem Dokument mitgeschickten EclEmma Coverage Report sind diese Daten und mehr in einem interaktiven Format zusammengefasst. 
Er erlaubt die eingehende Betrachtung der Überdeckung und ermöglicht es, die Abdeckung in den einzelnen Paketen, Klassen und Methoden detailliert aufzulisten.

\section{Integrationstest}

\subsection{Controller}

Für die Überprüfung der fehlerlosen Zusammenarbeit der voneinander abhängigen Systemkomponenten wurde eine Reihe weiterer Testfälle geschrieben. Dazu gehören vor allem Tests, die mittels eines TestRunners eine Instanz des Spiels beziehen, da dieser die komplette Applikation in einem „headless“ Modus hält. Dazu wurde die libGDX-Umgebung hinreichend nachgebaut. Viele der Controller konnten so bereits in einer nahezu \enquote{realen} Umgebung getestet werden, u.a.

\begin{description}
	\item[ProfileController und SettingController] Es wurde sichergestellt, dass der SettingController über einen Wechsel des Profils benachrichtigt wird und daraufhin entsprechende Anpassungen vorgenommen wurden.
	\item[ProfileController und StatisticController] Es wurde sichergestellt, das der StatisticController über einen Wechsel des Profils benachrichtigt wird und daraufhin entsprechende Anpassungen vorgenommen wurden.
\end{description}

\subsection{Spiel spielen}

Es wurden Integrationstests geschrieben, die das Zusammenspiel des GameControllers, RetroMan und der TiledMap sicherstellen.

\begin{description}
	\item[GameController, RetroMan und Level 1] Dieser Integrationstest stellt sicher, dass die grundsätzliche Logik des Spiels im ersten Level eingehalten wird. Unter anderem wurde so die \textit{Collision-Detection} und das Ablegen von Spielelementen getestet.
	\item[GameController, RetroMan und Level 6] Dieser Integrationstest erweitert die Abdeckung des ersten Tests, um eine bessere Überdeckung zu erhalten.
\end{description}

\section{Monkeyrunner}

Es wurden für alle Globalen Testfälle des Pflichtenheftes Skripte geschrieben, welche die Ausführung dieser auf einem Android-Testgerät mit spezifizierter Auflösung erlauben. Dazu muss die Applikation gestartet sein und sich im Hauptmenü befinden.

\section{Monkey Testing}

Mithilfe von Android konnte unsere App auf einem Testgerät installiert werden, welches dann über einen Befehl auf der Konsole eine gewisse Anzahl an Klicks auf dem Gerät ausgeführt hat. Hierbei wurden alle Bereiche unseres Programms tangiert und es konnte kein Fehlverhalten von diesem festgestellt werden. Da diese Eingaben sehr schnell erfolgten, ist auch sichergestellt, dass unser Programm dieser Belastung standhält.

\chapter{Gefundene und behobene Bugs}

\section{Löschen des aktiven Profils}
\begin{description}
	\item[Fehlersymptom] Falls ein Benutzer das aktive Profil gelöscht hat und im Anschluss ein neues erstellt hat, so hat das neue Profil die Statistiken und Einstellungen des vorherigen Profils übernommen.
	\item[Fehlergrund] Die Einstellungen und Statistiken des Profils wurden nicht korrekt gelöscht.
	\item[Fehlerbehebung] Es wurden entsprechende Aufrufe an die \enquote{destroy} Methoden der jeweiligen Klassen eingefügt.
\end{description}

\section{Fehler beim Starten eines Levels}
\begin{description}
	\item[Fehlersymptom] Es war möglich das Spiel zu nutzen, obwohl dem Spieler noch Tutorialhinweise angezeigt wurden.
	\item[Fehlergrund] Das entsprechende Attribut, was eine Sperre der Interaktion bewirkt, wurde nicht gesetzt, wenn ein Tutorialhinweis angezeigt wird.
	\item[Fehlerbehebung] Das entsprechende Attribut wird nun gesetzt und auch beim Beenden der Tutorialhinweise entfernt.
\end{description}

\section{Erstellen von Profilen mit unterschiedlicher Groß- und Kleinschreibung}
\begin{description}
	\item[Fehlersymptom] Das Erstellen eines zweiten Profils, welches sich nur durch Groß- und Kleinschreibung von einem anderen Profils unterschied, war möglich.
	\item[Fehlergrund] Es war keine Überprüfung eines neuen Profilnamens auf diese Eigenschaft implementiert.
	\item[Fehlerbehebung] Die Überprüfung der Methode, die die Erstellung eines Profils erlaubt, wurde verbessert.
\end{description}

\chapter{Testwerkzeuge}

\section{EclEmma}

EclEmma erlaubte es uns, die Codecoverage unseres Projekts zu überprüfen. Darüber hinaus verfügt dieses Tool über die Möglichkeit eines HTML-Exports.

\section{jUnit}

jUnit erlaubt das Testen einzelner Bestandteile unseres Codes. Außerdem waren wir durch jUnit in der Lage verschiedene Integrationstests auszuführen.

\section{monkeyrunner}

Zum Android-SDK gehört der sogenannte \enquote{monkeyrunner}. Dieser erlaubt es durch Skripte eine kontrollierte Abfolge von Eingaben auf einem Android Gerät auszuführen.

\section{adb shell monkey}

Darüber hinaus bietet die Android-Developer-Bridge (kurz adb) die Möglichkeit eine Vielzahl an Eingaben auf einem Android Gerät auszuführen. Hierbei kann spezifiziert werden, um welche Art von Eingaben es sich handelt, auf welche App sich die Eingaben konzentrieren und wieviele Eingaben ausgeführt werden sollen.

\end{document}