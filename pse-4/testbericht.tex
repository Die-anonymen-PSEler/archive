\documentclass[parskip=full]{scrreprt}

\usepackage{helvet}
\usepackage[T1]{fontenc}
\usepackage[ngerman]{babel}
\usepackage[utf8]{inputenc}
\inputencoding{utf8}
\usepackage{hyperref}
\usepackage{mathtools}
\usepackage[xindy]{imakeidx}
\makeindex
\usepackage{enumitem}
\usepackage{graphicx}
\usepackage{cleveref}
\usepackage[autostyle=true,german=quotes]{csquotes}
\renewcommand{\familydefault}{\sfdefault}
\begin{document}


\title{Implementierungsbericht: RetroMachines, \\ RetroFactory}
\author{Luca Becker, Henrike Hardt,\\Larissa Schmid, Adrian Schulte,\\Maik Wiesner}
\date{1. Dezember 2014}
\maketitle
\thispagestyle{empty}

\clearpage

\thispagestyle{empty}
\tableofcontents
\thispagestyle{empty}

\clearpage
\setcounter{page}{1}

\chapter{Einleitung}

\enquote{RetroMachines}

\chapter{Verbesserungen}

\chapter{Codequalität}

\chapter{Unittests}

\chapter{Integrationstest}

\chapter{Systemtests}

\chapter{Überdeckung}

\chapter{Lasttests}

\chapter{Feldtest}

\chapter{Gefundene und behobene Bugs}

\begin{description}
	\item[Löschen des aktiven Profils:] Es wurde ein Fehler behoben der Auftrifft, falls der Benutzer das aktive Profil löscht.
\end{description}

\chapter{Testwerkzeuge}

\section{EclEmma}

EclEmma erlaubte es uns die Codecoverage unseres Projekt zu überprüfen. Darüber hinaus verfügt dieses Tool über die Möglichkeit eines HTML-Exports.

\end{document}
